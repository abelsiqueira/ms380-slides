\documentclass[10 pt]{beamer}
\usepackage[brazil]{babel}
\usepackage[T1]{fontenc}
\usepackage{ae}
\usepackage[utf8]{inputenc}
\usepackage{framed}
%\usepackage[dvipsnames]{color}
\usepackage{graphicx}
\usepackage{algorithm}
\usepackage{algpseudocode}
\usepackage{epsfig}
\usepackage{tikz}
\usepackage{amssymb, amsmath, amsfonts}
\bibliographystyle{plain}
%\topmargin		0 cm
%\hoffset		0 cm
%\voffset		0 cm
%\evensidemargin		0 cm
%\oddsidemargin		0 cm
%\setlength{\textwidth}{16 cm}
%\setlength{\textheight}{21 cm}

\colorlet{mybasecolor}[rgb]{teal}

\colorlet{mydarkestcolor}[rgb]{mybasecolor!20!black}
\colorlet{mydarkcolor}[rgb]{mybasecolor!60!black}
\colorlet{mynormalcolor}[rgb]{mybasecolor!90!black}
\colorlet{mylightcolor}[rgb]{mybasecolor!10!white}

\colorlet{mysidebarcolor}[rgb]{mydarkcolor}
\colorlet{mylogobg}[rgb]{mylightcolor}
\colorlet{mylogofg}[rgb]{mydarkestcolor}
\colorlet{mysidebarauthor}[rgb]{mylightcolor}
\colorlet{mysidebartitle}[rgb]{mylightcolor}
\colorlet{mysidebarsecfg}[rgb]{mydarkestcolor}
\colorlet{mysidebarsecbg}[rgb]{mylightcolor}
\colorlet{mysidebarsubsecfg}[rgb]{mydarkestcolor}
\colorlet{mysidebarsubsecbg}[rgb]{mylightcolor}
\colorlet{myframetitlefg}[rgb]{mylightcolor}
\colorlet{myframetitlebg}[rgb]{mydarkcolor}
\colorlet{myitemcolor}[rgb]{mynormalcolor}
\colorlet{myfontcolor}[rgb]{mydarkestcolor}
\colorlet{mytitlefg}[rgb]{mylightcolor}
\colorlet{mytitlebg}[rgb]{mydarkcolor}
\colorlet{myauthorfg}[rgb]{mylightcolor}
\colorlet{myauthorbg}[rgb]{mydarkcolor}

\useinnertheme{rounded}
\setbeamercolor{structure}{fg = myitemcolor}
%\useoutertheme[height=1 cm,width=1.6 cm]{sidebar}
\useoutertheme{smoothtree}

\setbeamercolor{palette secondary}{bg=mylogobg,fg=mylogofg} %cor do logo
\setbeamercolor{sidebar}{bg=mysidebarcolor} %Background do sidebar
\setbeamercolor{palette sidebar tertiary}{fg = mysidebarauthor} %Autor no sidebar
\setbeamercolor{palette sidebar quaternary}{fg = mysidebartitle} %Titulo no sidebar
\setbeamercolor{section in sidebar}{fg = mysidebarsecfg,bg = mysidebarsecbg} %A cor eh a media de fg e bg
\setbeamercolor{subsection in sidebar}{fg = mysidebarsubsecfg,bg = mysidebarsubsecbg} %A cor eh a media de fg e bg
\setbeamercolor{frametitle}{fg = myframetitlefg, bg = myframetitlebg} 
\setbeamercolor{title}{bg = mytitlebg,fg = mytitlefg}
\setbeamerfont{title}{series = \bf}
\setbeamercolor{author}{bg = myauthorbg,fg = myauthorfg}
\setbeamerfont{author}{series = \it}
\setbeamercolor{normal text}{bg = white, fg = myfontcolor}
%\setbeamercolor{normal text}{bg = black, fg = white}
%\setbeamerfont{normal text}{family=serif}
\usefonttheme[onlymath]{serif}
%\logo{\includegraphics[scale=0.35]{logo.png}}
%\logo{$\min f(x)$}

\newcommand{\makesection}[1]{\section[#1]{#1}}
\newcommand{\makesubsection}[1]{\subsection[#1]{#1}}

\algrenewcommand\algorithmicif{\textbf{se}}
\algrenewcommand\algorithmicthen{\textbf{então}}
\algrenewcommand\algorithmicend{\textbf{fim do}}
\algrenewcommand\algorithmicwhile{\textbf{enquanto}}
\algrenewcommand\algorithmicdo{\textbf{faça}}
\algrenewcommand\algorithmicelse{\textbf{senão}}

\title{ MS380 - Aula 1 \\
Conceitos Elementares, Escalas e Taxas de Crescimento}
\author{Abel Soares Siqueira \\
Laércio Luís Vendite}
\date{}

\newcommand{\myframe}[1]{
\begin{frame}
 \frametitle{\insertsection \qquad {\small \insertsubsection}}
#1
\end{frame}}
\newcommand{\myframetop}[1]{
\begin{frame}[t]
 \frametitle{\insertsection \qquad {\small \insertsubsection}}
#1
\end{frame}}
% \newcommand{\visiblelbl}[1]{\label{#1}{\color{red}{#1}}}
\newcommand{\visiblelbl}[1]{\label{#1}}
\newcommand{\spc}{\vspace{0.5 cm}}
\newcommand{\systemtwo}[1]{$
  \left\{\begin{array}{rcrcr} #1
  \end{array}\right.$}
\newcommand{\systemthree}[1]{$
  \left\{\begin{array}{rcrcrcr}
  #1
  \end{array}\right.$}

\begin{document}

\begin{frame}
 \titlepage
\end{frame}

\makesection{Truncamento e Arredondamento}

\myframe {
  \begin{itemize}
    \item<1-> Truncamento: \\
      \begin{align*}
        12,34534 & \approx 12,34 \\
        0,0001 & \approx 0 \\
        25,999 & \approx 25,99 \\
        25,001 & \approx 25
      \end{align*}
    \item<2-> Arredondamento: \\
      \begin{align*}
        12,34534 & \approx 12,35 \\
        0,0001 & \approx 0 \\
        25,999 & \approx 26 \\
        25,001 & \approx 25
      \end{align*}
  \end{itemize}
}

\makesection{Escalas}

\myframe {
\begin{itemize}
  \item O registro das ocorrências de um estudo científico necessita de um meio 
  de se representar os acontecimentos e fenômenos adequadamente, ou seja, 
  de registrar dados, o que é realizado através das chamadas "escalas numéricas". 
  \item Basicamente, trata-se de modos de expressar as qualidades e/ou quantidades das coisas.
  \item As escalas podem ser  essencialmente de quatro tipos: Nominal, Ordinal, Intervalar e Razão.
  \end{itemize}
}

\makesubsection{Escala Nominal}

\myframe {
  Escala Nominal
  \begin{itemize}
    \item É o nível mais elementar de representação, baseado no 
      agrupamento e classificação de elementos para a formação de conjuntos distintos.
    \item As observações são divididas em categorias segundo um ou mais de seus atributos
    \item Registros essencialmente qualitativos, referentes a qual o 
      tipo de sujeito, objeto e/ou acontecimento que foi detectado em cada caso. 
    \item Espécies biológicas, áreas da farmácia, substâncias na química etc.
  \end{itemize}
}

\makesubsection{Escala Ordinal}

\myframe {
  Escala Ordinal
  \begin{itemize}
    \item É a avaliação de um fenômeno em termos de onde ele se 
      situa dentro de um conjunto de patamares ordenados, 
      variando desde um patamar mínimo até um máximo. 
    \item Geralmente, designa-se os valores de uma escala ordinal 
      em termos de numerais, ranking ou rótulos, sendo estas apenas 
      maneiras diferentes de se expressar o mesmo tipo de dado. 
    \item Escala de dor, diagnóstico médico, QI, escala de dureza dos metais, etc.
  \end{itemize}
}

\makesubsection{Escala Intervalar}

\myframe {
  Escala Intervalar
  \begin{itemize}
    \item É uma forma quantitativa de registrar um fenômeno, medindo-o em 
      termos da sua intensidade específica, ou seja, posicionando-o com 
      relação a um valor conhecido arbitrariamente denominado como ponto zero. 
    \item Duas variações iguais em termos de medidas intervalares 
      necessariamente correspondem a variações iguais em termos do 
      valor do que está sendo medido. 
    \item Altitude, tempo, etc
  \end{itemize}
}

\makesubsection{Escala de Razão}

\myframe {
  Escala de Razão
  \begin{itemize}
    \item É a mais completa e sofisticada das escalas numéricas. 
      Ela é uma quantificação produzida a partir da identificação de 
      um ponto zero que é fixo e absoluto, representando, de fato, um 
      ponto de nulidade, ausência e/ou mínimo. 
    \item Uma unidade de medida é definida em termos da diferença entre 
      o ponto zero e uma intensidade conhecida. 
    \item Um aspecto importante a ser observado é o de que, nas escalas de 
      razão, um valor de "2" efetivamente indica uma quantidade duas vezes 
      maior do que a do valor "1", e assim por diante, o que não necessariamente 
      acontece nas demais escalas. 
    \item Peso do corpo, altura do indivíduo, volume, etc...
  \end{itemize}
}

\makesection{Taxas de Crescimento}

\makesubsection{Valor Atual - Valor Futuro}

\myframe {
  \begin{center}
  \begin{tikzpicture}
    \draw (-1,0) -- (4,0);
    \pause
    \draw[->,color=blue] (0,1)  -- (0,0) node[midway,left] {$V_A$} node[below] {$0$};
    \pause
    \draw[->,color=blue] (3,0) node[below] {$1$} -- (3,1) node[midway,left] {$V_F$};
    \pause
    \draw[color=red] (0.2,-0.2) -- (2.8,-0.2) node[midway,below] {taxa $r$};
  \end{tikzpicture}
  \onslide<5->{
  $$ V_F = V_A(1+r) \qquad \qquad V_A = \frac{V_F}{1+r}$$}
  \end{center}
}

\myframe {
  \begin{center}
    \begin{tabular}{c|r}
    {\bf Ano} & {\bf População} \\ \hline
     1940 & 41.236.315 \\
     1950 & 51.944.397 \\
     1960 & 70.191.370 \\
     1970 & 93.139.037 \\
     1980 & 119.002.706 \\
     1990 & 146.352.150 \\
     2000 & 169.544.443 \\
     2010 & 190.732.694
    \end{tabular}
  \end{center}
}

\myframe{
  {\bf Qual a taxa de crescimento de 1980 para 1990?}
  \begin{align*}
    \onslide<1->{P_{1980} & = 119.002.706} \\
    \onslide<2->{P_{1990} & = 146.352.150} \\
    \onslide<3->{P_{1990} & = P_{1980}(1+r)} \\
    \onslide<4->{1 + r & = \frac{P_{1990}}{P_{1980}}} \\
    \onslide<5->{r & = \frac{146.352.150}{119.002.706} - 1} \\
    \onslide<6->{r & \approx 0.2298 = 22.98\%} 
  \end{align*}
}

\myframe{
  {\bf Se a taxa de crescimento de 1990 para 2000 se mantesse a mesma
    da década anterior, qual seria a população de 2000?}
  \begin{align*}
    \onslide<1->{P_{1990} & = 146.352.150} \\
    \onslide<2->{r & = 0.2298 } \\
    \onslide<3->{P_{2000} & = P_{1990}(1+r) } \\
    \onslide<4->{P_{2000} & = 146.352.150\times(1+0.2298) } \\
    \onslide<5->{P_{2000} & = 179.983.874 } \\
  \end{align*}
}

\makesubsection{Múltiplas Taxas}

\myframe{
  \begin{center}
  \begin{tikzpicture}
    \draw (-1,0) -- (5,0);
    \pause
    \draw[->,color=blue] (0,1)  -- (0,0) node[midway,left] {$P_0$} node[below] {$0$};
    \pause
    \draw[->,color=blue] (1,1)  -- (1,0) node[midway,left] {$P_1$} node[below] {$1$};
    \pause
    \draw[->,color=blue] (2,1)  -- (2,0) node[midway,left] {$P_2$} node[below] {$2$};
    \pause
    \node at (3,0.5) {$\cdots$};
    \pause
    \draw[->,color=blue] (4,0) node[below] {$n$} -- (4,1) node[midway,left] {$P_n$};
    \pause
    \draw[color=red] (0.2,-0.2) -- (0.8,-0.2) node[midway,below] {$r_1$};
    \pause
    \draw[color=red] (1.2,-0.2) -- (1.8,-0.2) node[midway,below] {$r_2$};
  \end{tikzpicture}
    \begin{align*}
      \onslide<9->{P_1 & = P_0 (1 + r_1) } \\
      \onslide<10->{P_2 & = P_1 (1 + r_2) = P_0 (1 + r_1)(1 + r_2) } \\
      \onslide<11->{P_n & = P_0(1+r_1)(1+r_2)\dots(1+r_n) }
    \end{align*}
  \end{center}
}

\makesubsection{Taxa acumulada}

\myframe {
  \begin{center}
    Considere uma única taxa $r_a$ de $P_0$ a $P_n$
    \pause
    $$ P_n = P_0(1 + r_a) $$
    \pause
    Pela fórmula anterior
    $$ P_0(1 + r_a) = P_0(1+r_1)(1+r_2)\dots(1+r_n) $$
    \pause
    Logo, a taxa acumulada é
    $$ r_a = (1+r_1)(1+r_2)\dots(1+r_n) - 1 $$
  \end{center}
}

\makesubsection{Taxas equivalentes}

\myframe {
  \begin{center}
    Exemplos:
    \pause
    \begin{description}
      \item[$i_a$] Taxa anual
      \item[$i_m$] Taxa mensal
      \item[$i_t$] Taxa trimestral
      \item[$i_d$] Taxa diária
    \end{description}   \pause
    \begin{align*}
     (1 + i_a) & = (1 + i_m)^{12} \\
     (1 + i_a) & = (1 + i_t)^4 \\
     (1 + i_m) & = (1 + i_d)^{30}
    \end{align*}
  \end{center}
}

\myframe {
  Lembrando que a taxa de crescimento da década de 1980 para 1990 foi 22.98\%,
  qual a taxa de crescimento anual equivalente nesse período?
  \spc
  Como uma década tem 10 anos, as taxas $i_a$ anual e $i_D$ da década satisfazem
  $$ (1 + i_D) = (1 + i_a)^{10} $$
  \pause
  Daí
  $$ (1 + i_a) = (1 + i_D)^{1/10} $$
  \pause
  $$ i_a = (1 + i_D)^{1/10} - 1 $$
  \pause
  $$ i_a = (1.2298)^{1/10} - 1 $$
  \pause
  $$ i_a = 0.0209 $$
}

\end{document}

\documentclass[10 pt]{beamer}
\usepackage[brazil]{babel}
\usepackage[T1]{fontenc}
\usepackage{ae}
\usepackage[utf8]{inputenc}
\usepackage{framed}
%\usepackage[dvipsnames]{color}
\usepackage{graphicx}
\usepackage{algorithm}
\usepackage{algpseudocode}
\usepackage{epsfig}
\usepackage{tikz}
\usepackage{pgfplots}
\usetikzlibrary{plotmarks}
\usepackage{amssymb, amsmath, amsfonts}
\bibliographystyle{plain}
%\topmargin		0 cm
%\hoffset		0 cm
%\voffset		0 cm
%\evensidemargin		0 cm
%\oddsidemargin		0 cm
%\setlength{\textwidth}{16 cm}
%\setlength{\textheight}{21 cm}

\colorlet{mybasecolor}[rgb]{teal}

\colorlet{mydarkestcolor}[rgb]{mybasecolor!20!black}
\colorlet{mydarkcolor}[rgb]{mybasecolor!60!black}
\colorlet{mynormalcolor}[rgb]{mybasecolor!90!black}
\colorlet{mylightcolor}[rgb]{mybasecolor!10!white}

\colorlet{mysidebarcolor}[rgb]{mydarkcolor}
\colorlet{mylogobg}[rgb]{mylightcolor}
\colorlet{mylogofg}[rgb]{mydarkestcolor}
\colorlet{mysidebarauthor}[rgb]{mylightcolor}
\colorlet{mysidebartitle}[rgb]{mylightcolor}
\colorlet{mysidebarsecfg}[rgb]{mydarkestcolor}
\colorlet{mysidebarsecbg}[rgb]{mylightcolor}
\colorlet{mysidebarsubsecfg}[rgb]{mydarkestcolor}
\colorlet{mysidebarsubsecbg}[rgb]{mylightcolor}
\colorlet{myframetitlefg}[rgb]{mylightcolor}
\colorlet{myframetitlebg}[rgb]{mydarkcolor}
\colorlet{myitemcolor}[rgb]{mynormalcolor}
\colorlet{myfontcolor}[rgb]{mydarkestcolor}
\colorlet{mytitlefg}[rgb]{mylightcolor}
\colorlet{mytitlebg}[rgb]{mydarkcolor}
\colorlet{myauthorfg}[rgb]{mylightcolor}
\colorlet{myauthorbg}[rgb]{mydarkcolor}

\useinnertheme{rounded}
\setbeamercolor{structure}{fg = myitemcolor}
%\useoutertheme[height=1 cm,width=1.6 cm]{sidebar}
\useoutertheme{smoothtree}

\setbeamercolor{palette secondary}{bg=mylogobg,fg=mylogofg} %cor do logo
\setbeamercolor{sidebar}{bg=mysidebarcolor} %Background do sidebar
\setbeamercolor{palette sidebar tertiary}{fg = mysidebarauthor} %Autor no sidebar
\setbeamercolor{palette sidebar quaternary}{fg = mysidebartitle} %Titulo no sidebar
\setbeamercolor{section in sidebar}{fg = mysidebarsecfg,bg = mysidebarsecbg} %A cor eh a media de fg e bg
\setbeamercolor{subsection in sidebar}{fg = mysidebarsubsecfg,bg = mysidebarsubsecbg} %A cor eh a media de fg e bg
\setbeamercolor{frametitle}{fg = myframetitlefg, bg = myframetitlebg} 
\setbeamercolor{title}{bg = mytitlebg,fg = mytitlefg}
\setbeamerfont{title}{series = \bf}
\setbeamercolor{author}{bg = myauthorbg,fg = myauthorfg}
\setbeamerfont{author}{series = \it}
\setbeamercolor{normal text}{bg = white, fg = myfontcolor}
%\setbeamercolor{normal text}{bg = black, fg = white}
%\setbeamerfont{normal text}{family=serif}
\usefonttheme[onlymath]{serif}
%\logo{\includegraphics[scale=0.35]{logo.png}}
%\logo{$\min f(x)$}

\newcommand{\makesection}[1]{\section[#1]{#1}}
\newcommand{\makesubsection}[1]{\subsection[#1]{#1}}
\newcommand{\tikzcircle}{arc (0:360:0.01)}

\algrenewcommand\algorithmicif{\textbf{se}}
\algrenewcommand\algorithmicthen{\textbf{então}}
\algrenewcommand\algorithmicend{\textbf{fim do}}
\algrenewcommand\algorithmicwhile{\textbf{enquanto}}
\algrenewcommand\algorithmicdo{\textbf{faça}}
\algrenewcommand\algorithmicelse{\textbf{senão}}

\title{ MS380 - Aula 6 \\
Máximos, Mínimos e Gráficos}
\author{Abel Soares Siqueira \\
Laércio Luís Vendite}
\date{}

\newcommand{\myframe}[1]{
\begin{frame}
 \frametitle{\insertsection \qquad {\small \insertsubsection}}
#1
\end{frame}}
\newcommand{\myframetop}[1]{
\begin{frame}[t]
 \frametitle{\insertsection \qquad {\small \insertsubsection}}
#1
\end{frame}}
% \newcommand{\visiblelbl}[1]{\label{#1}{\color{red}{#1}}}
\newcommand{\visiblelbl}[1]{\label{#1}}
\newcommand{\spc}{\vspace{0.5 cm}}
\newcommand{\systemtwo}[1]{$
  \left\{\begin{array}{rcrcr} #1
  \end{array}\right.$}
\newcommand{\systemthree}[1]{$
  \left\{\begin{array}{rcrcrcr}
  #1
  \end{array}\right.$}

\newcommand{\limninf}{\lim_{n\rightarrow\infty}}
\newcommand{\limx}[1]{\lim_{x\rightarrow{#1}}}
\newcommand{\limh}[1]{\lim_{h\rightarrow{#1}}}
\newcommand{\fundef}[1]{\left\{\begin{array}{ll}{#1}\end{array}\right.}

\begin{document}

\begin{frame}
 \titlepage
\end{frame}

\makesection{Interpretação das Derivadas}

\myframe {
  Como visto, a derivada pode ser interpretada como a inclinação
  da reta tangente à função no ponto onde é calculada.

  A derivada também indica se a função está crescendo ou decrescendo
  naquele ponto (ou intervalo).

  Sendo assim, uma derivada nula, ou seja, uma reta horizontal, é um indicativo 
  que naquele ponto pode ocorrer um máximo ou um mínimo.

  \begin{center}
    \begin{tikzpicture}[domain=-1:1]
      \draw[blue] plot (\x,{\x*\x});
      \draw[red] (-1,0) -- (1,0);
    \end{tikzpicture}
    \hspace{1 cm}
    \begin{tikzpicture}[domain=-1:1]
      \draw[blue] plot (\x,{-\x*\x});
      \draw[red] (-1,0) -- (1,0);
    \end{tikzpicture}
  \end{center}
}

\myframe {
  Uma maneira de encontrar o máximo ou mínimo de uma função então é
  procurar pelos pontos onde a derivada é nula.

  Note, no entanto, que isso não é suficiente. Pois um ponto com a
  derivada nula pode ser um máximo, um mínimo, ou nenhum dos dois.
  
  $$ f(x) = x^3, \qquad f'(x) = 3x^2, \qquad f'(0) = 0 $$
  \begin{center}
    \begin{tikzpicture}[domain=-1:1]
      \draw[blue] plot (\x,{\x*\x*\x});
      \draw[red] (-1,0) -- (1,0);
    \end{tikzpicture}
  \end{center}
}

\myframe {
  A segunda derivada de uma função indica a concavidade de uma função.

  Se a segunda derivada é positiva no ponto, a concavidade é para cima.
  Se a segunda derivada é negativa, a concavidade é para baixo.

  Com a segunda derivada, podemos verificar se o ponto que encontramos
  é um ponto de máximo ou mínimo da função.
  Note que ainda existe a possibilidade de não ser possível identificar
  que tipo de ponto temos.

  \begin{itemize}
    \item Se $f'(\barra{x}) = 0$ e $f''(\barra{x}) > 0$, então o ponto
      é um minimizador da função $f$.
    \item Se $f'(\barra{x}) = 0$ e $f''(\barra{x}) < 0$, então o ponto
      é um maximizador da função $f$.
  \end{itemize}
}

\subsection{Exemplos}

\myframe {
  Considere a função $f(x) = ax^2 + bx + c$, com $a\neq 0$.
  A primeira e segunda derivadas dessa função são
  $$ f'(x) = 2ax + b \qquad \mbox{e} \qquad
     f''(x) = 2a. $$
  Veja que a concavidade de $f$ no ponto $x$ é $2a$.
  A concavidade é constante, é tem o mesmo sinal do coeficiente
  quadrático, como já esperávamos.

  Fazendo
  $$ f'(\barra{x}) = 0, $$
  temos
  $$ 2a\barra{x} + b = 0 \qquad \Longrightarrow \qquad
    \barra{x} = -\frac{b}{2a} $$
  Se $a > 0$, esse ponto é um minimizador, e se $a < 0$,
  esse ponto é um maximizador.
}

\myframe {
  {\bf Resolva:} Um tipógrafo quer imprimir um boletim com
  $72 \mbox{cm}^2$ de texto impresso, com margens inferior
  e superior de $6$ cm e laterais de $3$ cm. Quais as dimensões
  das folhas para conseguir a maior economia?

  \begin{minipage}{3.5 cm}
  \begin{tikzpicture}[scale=0.7]
    \draw (0,0) rectangle (4,5);
    \draw[fill=yellow] (0.6,0.8) rectangle (3.3,4.2);
    \draw[<->] (0,5/2) -- node[below] {$3$} (0.6,5/2);
    \draw[<->] (4/2,0) -- node[left] {$6$} (4/2,0.8);
    \draw[<->,blue] (0.6,4.4) -- node[above] {$x$} (3.3,4.4);
    \draw[<->,blue] (3.5,0.8) -- node[right] {$y$} (3.5,4.2);
  \end{tikzpicture}
  \end{minipage}
  \begin{minipage}{6 cm}
    $$A = (x+3+3)(y+6+6) = (x+6)(y+12) $$
    $$xy = 72 \Rightarrow y = \dfrac{72}{x}$$
    \begin{align*}
      A(x) & = (x+6)(\frac{72}{x} + 12) = 144 + \frac{432}{x} + 12x \\
      A'(x) & = -\frac{432}{x^2} + 12
    \end{align*}
  \end{minipage}
  $$ A'(\barra{x}) = 0 \Rightarrow 12 = \frac{432}{x^2} \Rightarrow
    x = \sqrt{\frac{432}{12}} = 6 \Rightarrow y = 12 $$
    A folha deve ter $12 \mbox{cm}\times 24\mbox{cm}$.
}

\subsection{Assíntotas}

\myframe {
  As assíntotas horizontais de uma função são as retas que indicam
  para onde a função tende quando converge quando $x\rightarrow\infty$,
  e para $x\rightarrow-\infty$.

  Exemplo: $f(x) = \dfrac{x+1}{x}$.
  $$ \limx{+\infty}\frac{x+1}{x} = 1 \quad
      \Longrightarrow \quad \mbox{$f$ tem uma assíntota horizontal em } 
        y=1, x\rightarrow+\infty$$
  $$ \limx{-\infty}\frac{x+1}{x} = 1 \quad
      \Longrightarrow \quad \mbox{$f$ tem uma assíntota horizontal em } 
        y=1, x\rightarrow-\infty$$
}

\myframe {
  As assíntotas verticais são pontos onde acontece ao menos um dos seguintes:
  $$ \limx{a+} = \pm\infty \qquad \limx{a-} = \pm\infty $$
  Exemplo: $f(x) = \dfrac{x+1}{x}$.
  $$ \limx{0+}\frac{x+1}{x} = +\infty $$
    assíntota vertical em $x = 0$, com $f(x)\rightarrow+\infty$, pela direita.
  $$ \limx{0-}\frac{x+1}{x} = -\infty $$
    assíntota vertical em $x = 0$, com $f(x)\rightarrow-\infty$, pela esqueda.
}

\myframe {
  As assíntotas inclinadas são retas $y = ax+b$ tais que
  $$ \limx{\infty}[f(x) - ax-b] = 0, $$
  ou similar para $-\infty$.

  Todas as funções racionais com o grau do polinômio do numerador um a mais
  que o grau do polinômio do denominador tem assíntotas inclinadas.
  Nesse caso, para encontrar a assíntota basta fazer divisão de polinômios.
}

\myframe {
  Exemplo: $f(x) = \dfrac{x^2}{x+1}$. Fazendo a divisão, temos
  $$ f(x) = x-1 + \frac{1}{x+1} $$
  A assíntota é $y = x-1$. Basta ver que
  $$ \limx{\infty}[f(x)-(x-1)] = \limx{\infty}\frac{1}{x+1} = 0 $$
  Similar para $x \rightarrow-\infty$.
}

\myframe {
  Outra maneira de encontrar a assíntota inclinada é calcular o limite
  da derivada para encontrar a inclinação, e o limite da diferença
  entre a função e a reta com essa inclinação para encontrar o termo
  independente.
}

\myframe {
  Exemplo: $f(x) = \sqrt{x^2 + 2x + 2}$. Inicialmente, note que o domínio
  máximo de $f$ é $\R$ (por quê?).
  $$ f'(x) = \frac{2x + 2}{2\sqrt{x^2 + 2x + 2}} = \frac{x+1}{\sqrt{x^2+2x+2}} $$
  $$ \limx{+\infty}f'(x) = 
    \limx{+\infty}\frac{1+\frac{1}{x}}{\sqrt{1+\frac{2}{x}+\frac{2}{x^2}}} = 1$$
  Então $a = 1$. Agora fazemos o limite de $f(x) - ax$.
}

\myframe {
  \begin{align*}
    \limx{+\infty} [\sqrt{x^2+2x+2} - x] 
      & = \limx{+\infty} \frac{2x+2}{\sqrt{x^2+2x+2}+x} \\
      & = \limx{+\infty} \frac{2+\frac{2}{x}} 
          {\sqrt{1 + \frac{2}{x} + \frac{2}{x^2}} + 1} \\
      & = \frac{2}{\sqrt{1}+1} = 1
  \end{align*}
  Então $b = 1$. Portanto, a assíntota é $y = x + 1$.
}

\subsection{Simetria}

\myframe {
  Se $f(-x) = f(x)$, a função $f$ é dita par, e seu gráfico é simétrico
  pelo eixo $y$.
  
  {\bf Exemplo:} $f(x) = x^2, f(x) = \dfrac{x^2}{x^2 + 1}, f(x) = \cos(x)$.
  \spc

  Se $f(-x) = -f(x)$, a função $f$ é dita ímpar, e seu gráfico é simétrico
  pela origem.

  {\bf Exemplo:} $f(x) = x^3, f(x) = \dfrac{x}{x^2 + 1}, f(x) = \sin(x)$.
  \spc

  Se $f(x+T) = f(x)$, então a função $f$ é dita periódica, e o seu
  período é o menor valor $T$ que satisfaz a relação.
  Seu gráfico se repete a cada intervalo de tamanho $T$.

  {\bf Exemplo:} $f(x) = \tan(x), f(x) = \sin(x)\cos^2(x)$.
}

\subsection{Gráficos}

\myframe {
  Para fazer o gráfico de uma função, procure seguir os seguintes pontos:
  \begin{itemize}
    \item Domínio máximo
    \item Intersecções nos eixos
    \item Simetria
    \item Assíntotas
    \item Pontos críticos
    \item Intervalos de crescimento
    \item Concavidade
  \end{itemize}
}

\myframe {
  {\bf Exemplo: } $f(x) = xe^x$, $f'(x) = e^x(1+x)$, $f''(x) = e^x(2+x)$.
  \begin{itemize}
    \item O domínio de $f$ é $\R$.
    \item $f(x) = 0 \Rightarrow x = 0$, e $f(0) = 0$.
    \item $f$ não tem nenhuma simetria.
    \item $\displaystyle \limx{\infty}f(x) = +\infty$ e
      $$ \displaystyle \limx{-\infty}xe^x =
        \limx{-\infty}\frac{x}{e^{-x}} \overset{L'H}{=}
        \limx{-\infty}\frac{1}{-e^{-x}} =
        \limx{-\infty}-e^x = 0 $$
     $$ \limx{\infty} f'(x) = +\infty $$
     Só tem uma assíntota horizontal para $x\rightarrow-\infty$, que
      será o eixo $x$.
    \item $f'(x) = 0 \Rightarrow x = -1$, e $f''(-1) = e^{-1} > 0$, então
      $f$ tem um minimizador em $x=-1$. $f(-1) = -e^{-1}$.
    \item $f'(x) = e^x(1+x)$. $e^x$ é sempre positivo, então o sinal depende
      apenas de $(1+x)$. Então se $x > -1$, $f'(x) > 0$, e $f$ é crescente,
      e se $x < -1$, $f'(x) < 0$, e $f$ é decrescente.
  \end{itemize}
}

\myframe {
  \begin{itemize}
    \item $f''(x) = e^x(2+x)$. Pelo mesmo argumento anterior,
      se $x > -2$, $f''(x) > 0$, e $f$ é côncava para cima, e se
      $x < -2$, $f''(x) < 0$, e $f$ é côncava para baixo. Note que
      $x = -2$ é um ponto de inflexão. $f(-2) = -2e^{-2}$.
  \end{itemize}
  \begin{center}
  \begin{tikzpicture}[domain=-4:1]
    \draw[->] (-4.1,0) -- (1.1,0) node[right] {$x$};
    \draw[->] (0,-1.1) -- (0,3.1) node[left] {$y$};
    \draw[blue] plot (\x, {\x*exp(\x)});
    \draw (-1,0.05) node[above] {$-1$} -- (-1,-0.05);
    \draw (-2,0.05) node[above] {$-2$} -- (-2,-0.05);
    \draw (-0.05,-1/2.72) -- (0.05,-1/2.72) node[right] {$-e^{-1}$};
    \draw[dashed] (-1,0) -- (-1,-1/2.72) -- (0,-1/2.72);
  \end{tikzpicture}
  \end{center}
}

\myframe {
  {\bf Exemplo: } $f(x) = \sqrt{x^2 + 1}$, $f'(x) = \dfrac{x}{\sqrt{x^2 + 1}}$,
    $f''(x) = \dfrac{1}{(x^2 + 1)^{3/2}}$.
  \begin{itemize}
    \item O domínio máximo de $f$ é $\R$.
    \item Não existe interseção com o eixo $x$, e a intersecção com o eixo $y$ é
      $f(0) = 1$.
    \item $f(-x) = \sqrt{(-x)^2 + 1} = \sqrt{x^2 + 1} = f(x)$. Então $f$ é
      par, e seu gráfico é simétrico em relação ao eixo $y$. Sendo assim, 
      precisamos analisar somente a parte positiva do gráfico.
    \item $\displaystyle\limx{+\infty} f(x) = +\infty$, então o gráfico não tem
      assíntotas horizontais. 
      $$\limx{+\infty} f'(x) = \limx{+\infty}\frac{1}{\sqrt{1 + \frac{1}{x^2}}}
        = 1$$
      Então o gráfico tem uma assíntota inclinada com inclinação $a = 1$.
  \end{itemize}
}

\myframe {
  \begin{align*}
    \limx{\infty} [f(x) - x] & = \limx{\infty} [\sqrt{x^2 + 1} - x] \\
      & = \limx{\infty} \frac{x^2+1-x^2}{\sqrt{x^2+1} + x} \\
      & = \limx{\infty} \frac{1}{\sqrt{x^2 + 1}+x} = 0
  \end{align*}
    Então a assíntota é $y = x$.
  \begin{itemize}
    \item $f'(x) = 0 \Rightarrow x = 0$. $f''(0) = 1 > 0$, então
      $x = 0$ é um minimizador de $f$, e $f(0) = 1$.
    \item Para $x > 0$, $f(x) > 0$. Então $f$ é crescente.
    \item $f(x) > 0$ para todo $x \in \R$. Então a concavidade é sempre para cima.
  \end{itemize}
}

\myframe {
  \begin{center}
  \begin{tikzpicture}[domain=-8:8,scale=0.5]
    \draw[->] (-8.1,0) -- (8.1,0);
    \draw[->] (0,-0.1) -- (0,9.1);
    \draw[blue] plot (\x, {sqrt(\x*\x + 1)});
    \draw[domain=2:8,gray,dashed] plot (\x,\x);
    \draw[domain=-2:-8,gray,dashed] plot (\x,-\x);
    \draw (-0.1,1) -- (0.1,1) node[below left] {$1$};
  \end{tikzpicture}
  \end{center}
}

\end{document}

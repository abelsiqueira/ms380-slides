\documentclass[10 pt]{beamer}
\usepackage[brazil]{babel}
\usepackage[T1]{fontenc}
\usepackage{ae}
\usepackage[utf8]{inputenc}
\usepackage{framed}
%\usepackage[dvipsnames]{color}
\usepackage{graphicx}
\usepackage{algorithm}
\usepackage{algpseudocode}
\usepackage{epsfig}
\usepackage{tikz}
\usepackage{pgfplots}
\usetikzlibrary{plotmarks}
\usepackage{amssymb, amsmath, amsfonts}
\bibliographystyle{plain}
%\topmargin		0 cm
%\hoffset		0 cm
%\voffset		0 cm
%\evensidemargin		0 cm
%\oddsidemargin		0 cm
%\setlength{\textwidth}{16 cm}
%\setlength{\textheight}{21 cm}

\colorlet{mybasecolor}[rgb]{teal}

\colorlet{mydarkestcolor}[rgb]{mybasecolor!20!black}
\colorlet{mydarkcolor}[rgb]{mybasecolor!60!black}
\colorlet{mynormalcolor}[rgb]{mybasecolor!90!black}
\colorlet{mylightcolor}[rgb]{mybasecolor!10!white}

\colorlet{mysidebarcolor}[rgb]{mydarkcolor}
\colorlet{mylogobg}[rgb]{mylightcolor}
\colorlet{mylogofg}[rgb]{mydarkestcolor}
\colorlet{mysidebarauthor}[rgb]{mylightcolor}
\colorlet{mysidebartitle}[rgb]{mylightcolor}
\colorlet{mysidebarsecfg}[rgb]{mydarkestcolor}
\colorlet{mysidebarsecbg}[rgb]{mylightcolor}
\colorlet{mysidebarsubsecfg}[rgb]{mydarkestcolor}
\colorlet{mysidebarsubsecbg}[rgb]{mylightcolor}
\colorlet{myframetitlefg}[rgb]{mylightcolor}
\colorlet{myframetitlebg}[rgb]{mydarkcolor}
\colorlet{myitemcolor}[rgb]{mynormalcolor}
\colorlet{myfontcolor}[rgb]{mydarkestcolor}
\colorlet{mytitlefg}[rgb]{mylightcolor}
\colorlet{mytitlebg}[rgb]{mydarkcolor}
\colorlet{myauthorfg}[rgb]{mylightcolor}
\colorlet{myauthorbg}[rgb]{mydarkcolor}

\useinnertheme{rounded}
\setbeamercolor{structure}{fg = myitemcolor}
%\useoutertheme[height=1 cm,width=1.6 cm]{sidebar}
\useoutertheme{smoothtree}

\setbeamercolor{palette secondary}{bg=mylogobg,fg=mylogofg} %cor do logo
\setbeamercolor{sidebar}{bg=mysidebarcolor} %Background do sidebar
\setbeamercolor{palette sidebar tertiary}{fg = mysidebarauthor} %Autor no sidebar
\setbeamercolor{palette sidebar quaternary}{fg = mysidebartitle} %Titulo no sidebar
\setbeamercolor{section in sidebar}{fg = mysidebarsecfg,bg = mysidebarsecbg} %A cor eh a media de fg e bg
\setbeamercolor{subsection in sidebar}{fg = mysidebarsubsecfg,bg = mysidebarsubsecbg} %A cor eh a media de fg e bg
\setbeamercolor{frametitle}{fg = myframetitlefg, bg = myframetitlebg} 
\setbeamercolor{title}{bg = mytitlebg,fg = mytitlefg}
\setbeamerfont{title}{series = \bf}
\setbeamercolor{author}{bg = myauthorbg,fg = myauthorfg}
\setbeamerfont{author}{series = \it}
\setbeamercolor{normal text}{bg = white, fg = myfontcolor}
%\setbeamercolor{normal text}{bg = black, fg = white}
%\setbeamerfont{normal text}{family=serif}
\usefonttheme[onlymath]{serif}
%\logo{\includegraphics[scale=0.35]{logo.png}}
%\logo{$\min f(x)$}

\newcommand{\makesection}[1]{\section[#1]{#1}}
\newcommand{\makesubsection}[1]{\subsection[#1]{#1}}
\newcommand{\tikzcircle}{arc (0:360:0.01)}
\newcommand{\dif}{\,\mathrm{d}}

\algrenewcommand\algorithmicif{\textbf{se}}
\algrenewcommand\algorithmicthen{\textbf{então}}
\algrenewcommand\algorithmicend{\textbf{fim do}}
\algrenewcommand\algorithmicwhile{\textbf{enquanto}}
\algrenewcommand\algorithmicdo{\textbf{faça}}
\algrenewcommand\algorithmicelse{\textbf{senão}}

\title{ MS380 - Aula 7 \\
Integral}
\author{Abel Soares Siqueira \\
Laércio Luís Vendite}
\date{}

\newcommand{\myframe}[1]{
\begin{frame}
 \frametitle{\insertsection \qquad {\small \insertsubsection}}
#1
\end{frame}}
\newcommand{\myframetop}[1]{
\begin{frame}[t]
 \frametitle{\insertsection \qquad {\small \insertsubsection}}
#1
\end{frame}}
% \newcommand{\visiblelbl}[1]{\label{#1}{\color{red}{#1}}}
\newcommand{\visiblelbl}[1]{\label{#1}}
\newcommand{\spc}{\vspace{0.5 cm}}
\newcommand{\systemtwo}[1]{$
  \left\{\begin{array}{rcrcr} #1
  \end{array}\right.$}
\newcommand{\systemthree}[1]{$
  \left\{\begin{array}{rcrcrcr}
  #1
  \end{array}\right.$}

\newcommand{\limninf}{\lim_{n\rightarrow\infty}}
\newcommand{\limx}[1]{\lim_{x\rightarrow{#1}}}
\newcommand{\limh}[1]{\lim_{h\rightarrow{#1}}}
\newcommand{\limite}[2]{\lim_{{#1}\rightarrow{#2}}}
\newcommand{\fromto}[2]{\bigg|_{#1}^{#2}}
\newcommand{\fundef}[1]{\left\{\begin{array}{ll}{#1}\end{array}\right.}

\begin{document}

\begin{frame}
 \titlepage
\end{frame}

\makesection{Integral indefinida - Primitivas}

\myframe{
  Dada a funçào $f(x)$, queremos determinar a função $F(x)$, cuja derivada
  é igual à função dada:
  $$ F'(x) = f(x) $$

  Definimos como {\bf primitiva} esta função $F(x)$.

  Exemplos:
  \begin{itemize}
    \item Se $f(x) = x$, então $F(x) = \frac{x^2}{2}$.
    \item Se $f(x) = \cos x$, então $F(x) = \sin x$.
  \end{itemize}

  Note que, para $f(x) = \cos x$, então $G(x) = \sin x + 5$ também tem
  sua derivada igual a $f(x)$. Então $G$ também é uma primitiva para $f$.
  Portanto, uma função admite mais de uma primitiva.
}

\myframe{
  Considere
  $$ f(x) = x^2 $$
  As funções
  $$ F(x) = \frac{x^3}{3} + 1 \qquad
     G(x) = \frac{x^3}{3} - 3 \qquad
     H(x) = \frac{x^3}{3} + 5 $$
     são primitivas de $f$.
  De uma maneira geral, as primitivas são
  $$ F(x) = \frac{x^3}{3} + C, $$
  onde $C$ é uma constante.
}

\myframe{
  Denomina-se {\bf Integral indefinida} de uma função $f(x)$ a operação
  de determinação da expressão da primitiva dessa função.
  Esta operação é representada por
  $$ \int\! f(x)\dif x $$
  Então, 
  $$ \int\! f(x)\dif x = F(x) + C, $$
  com $F'(x) = f(x)$.
}

\makesubsection{Exemplos}

\myframe{

    $$F(x) = \frac{x^2}{2} \qquad 
      f(x) = F'(x) = x \qquad 
      \int\! x\dif x = \frac{x^2}{2} + C$$ \\
    $$F(x) = \frac{x^{10}}{10} \qquad 
      f(x) = F'(x) = x^9 \qquad 
      \int\! x^9\dif x = \frac{x^9}{10} + C$$ \\
    $$F(x) = e^x \qquad 
      f(x) = F'(x) = e^x \qquad 
      \int\! e^x \dif x = e^x + C$$
    $$F(x) = \cos x \qquad
      f(x) = F'(x) = -\sin x \qquad
      \int\! (-\sin x) \dif x = \cos x + C$$
    $$F(x) = \ln x \qquad
      f(x) = F'(x) = \frac{1}{x} \qquad
      \int\! \frac{1}{x}\dif x = \ln x + C$$
}

\makesubsection{Propriedades}

\myframe{
  $$ \int\![f(x) + g(x)]\dif x = \int\! f(x)\dif x + \int\! g(x)\dif x $$
  $$ \int\!(x^2 + 2x + 1)\dif x = \int\! x^2\dif x + \int\! 2x\dif x + \int\! \dif x $$
  \vspace{1cm}
  $$ \int\! af(x) \dif x = a\int\! f(x) \dif x $$
  $$ \int\! 5x^2 \dif x = 5\int\! x^2 \dif x $$
}

\makesection{Integral definida}

\myframe{
  \begin{center}
  \begin{tikzpicture}[domain=0.5:6.5]
    \draw[->] (-1,0) -- (7,0);
    \draw[->] (0,-1) -- (0,5);
    \draw[gray] (1,-0.1)node[below,black]{$a$} -- (1,5);
    \draw[gray] (6,-0.1)node[below,black]{$b$} -- (6,5);
    \only<1> {
    \foreach \x in {1,3.5,...,5.999}{
      \draw[fill=lightgray] (\x,0) -- (\x,{1+0.1*(1-\x-2.5)*(\x+2.5-12)}) 
        -- (\x+2.5,{1+0.1*(1-\x-2.5)*(\x+2.5-12)}) -- (\x+2.5,0);
      \draw[fill=gray] (\x,0) -- (\x,{1+0.1*(1-\x)*(\x-12)}) 
        -- (\x+2.5,{1+0.1*(1-\x)*(\x-12)}) -- (\x+2.5,0);
      \draw[<->] (3.5,-0.2) -- node[below]{$h$} (3.5+2.5,-0.2);
    }}
    \only<2> {
    \foreach \x in {1,2,...,5.999}{
      \draw[fill=lightgray] (\x,0) -- (\x,{1+0.1*(0-\x)*(\x-11)}) 
        -- (\x+1,{1+0.1*(0-\x)*(\x-11)}) -- (\x+1,0);
      \draw[fill=gray] (\x,0) -- (\x,{1+0.1*(1-\x)*(\x-12)}) 
        -- (\x+1,{1+0.1*(1-\x)*(\x-12)}) -- (\x+1,0);
      \draw[<->] (3,-0.2) -- node[below]{$h$} (3+1,-0.2);
    }}
    \only<3> {
    \foreach \x in {1,1.5,...,5.999}{
      \draw[fill=lightgray] (\x,0) -- (\x,{1+0.1*(1-\x-1/2)*(\x-12+1/2)}) 
        -- (\x+1/2,{1+0.1*(1-1/2-\x)*(\x-12+1/2)}) -- (\x+1/2,0);
      \draw[fill=gray] (\x,0) -- (\x,{1+0.1*(1-\x)*(\x-12)}) 
        -- (\x+1/2,{1+0.1*(1-\x)*(\x-12)})-- (\x+1/2,0);}
      \draw[<->] (3,-0.2) -- node[below]{$h$} (3+1/2,-0.2);
    }
    \only<4> {
    \foreach \x in {1,1.333333,...,5.999}{
      \draw[fill=lightgray] (\x,0) -- (\x,{1+0.1*(1-\x-1/3)*(\x-12+1/3)}) 
        -- (\x+1/3,{1+0.1*(1-1/3-\x)*(\x-12+1/3)}) -- (\x+1/3,0);
      \draw[fill=gray] (\x,0) -- (\x,{1+0.1*(1-\x)*(\x-12)}) 
        -- (\x+1/3,{1+0.1*(1-\x)*(\x-12)}) -- (\x+1/3,0);}
      \draw[<->] (3,-0.2) -- node[below]{$h$} (3+1/3,-0.2);
    }
    \only<5> {
    \foreach \x in {1,1.25,...,5.999}{
      \draw[fill=lightgray] (\x,0) -- (\x,{1+0.1*(1-\x-1/4)*(\x-12+1/4)}) 
        -- (\x+1/4,{1+0.1*(1-1/4-\x)*(\x-12+1/4)}) -- (\x+1/4,0);
      \draw[fill=gray] (\x,0) -- (\x,{1+0.1*(1-\x)*(\x-12)}) 
        -- (\x+1/4,{1+0.1*(1-\x)*(\x-12)}) -- (\x+1/4,0);}
      \draw[<->] (3,-0.2) -- node[below]{$h$} (3+1/4,-0.2);
    }
    \draw[thick,blue] plot (\x,{1+0.1*(1-\x)*(\x-12)});
  \end{tikzpicture}
  \end{center}
}

\myframe{
  Consideramos $f$ crescente. Uma expansão similar pode ser feita para $f$
    descrescente, e uma função geral pode ser separada em intervalos.
  $$ h = \frac{b-a}{N} $$
  \vspace{0.5cm}
  $$ s_N = \sum_{n=0}^{N-1} f(a + kh)h
    = \sum_{n=0}^{N-1} f\bigg(a + \frac{k(b-a)}{N}\bigg)\bigg(\frac{b-a}{N}\bigg) $$
  $$ S_N = \sum_{n=1}^{N} f(a + kh)h
    = \sum_{n=1}^{N} f\bigg(a + \frac{k(b-a)}{N}\bigg)\bigg(\frac{b-a}{N}\bigg) $$
}

\myframe{
  $$ s_N \leq \int_a^b\! f(x)\dif x \leq S_N $$
  $$ \limite{N}{\infty} s_N \leq \int_a^b\! f(x)\dif x
    \leq \limite{N}{\infty} S_N $$
  \vspace{0.5cm}

  Se $F$ é uma primitiva de $f$, então
  $$ \int_a^b\!f(x)\dif x = F(b) - F(a) = F(x)\fromto{a}{b} $$
}

\makesubsection{Exemplos}

\myframe{
  $$ \int_0^1\!x\dif x = \frac{x^2}{2}\fromto{0}{1} =
    \frac{1^2}{2} - \frac{0^2}{2} = \frac{1}{2} $$
  $$ \int_2^5\!x^3\dif x = \frac{x^4}{4}\fromto{2}{5} =
    \frac{5^4}{4} - \frac{2^4}{4} = \frac{609}{4} $$
  $$ \int_0^{\frac{\pi}{4}}\!2\cos x = 2\sin x\fromto{0}{\frac{\pi}{4}} =
    2\sin\bigg(\frac{\pi}{4}\bigg) - 2\sin(0) = \sqrt{2}$$
  $$ \int_1^e\!\frac{1}{x}\dif x = \ln x\fromto{1}{e} =
    \ln e - \ln 1 = 1 $$
  $$ \int_{-1}^1\!e^x\dif x = e^x\fromto{-1}{1} =
    e - e^{-1} = \frac{e^2-1}{e}$$
}

\makesubsection{Propriedades}

\myframe{
  $$ \int_a^b\!f(x)\dif x = -\int_b^a\!f(x)\dif x $$
  \vspace{0.5cm}

  Se $f(x) \leq g(x)$ no intervalo $(a,b)$, então
  $$ \int_a^b\!f(x)\dif x \leq \int_a^b\!g(x)\dif x $$

  \vspace{0.5cm}
  Se $f(x)$ é limitada inferiormente por $m$ e superiormente por $M$, então
  $$ m(b-a) \leq \int_a^b\!f(x)\dif x \leq M(b-a) $$
}

\makesection{Funções Tradicionais}

\myframe{
  $$ \int\!x^n\dif x = \frac{x^{n+1}}{n+1} + C, \qquad n\neq-1 $$
  $$ \int\!\sin(ax)\dif x = -\frac{\cos(ax)}{a} + C, \qquad a \neq 0 $$
  $$ \int\!\cos(ax)\dif x = \frac{\sin(ax)}{a} + C, \qquad a \neq 0 $$
  $$ \int\!e^{ax}\dif x = \frac{e^{ax}}{a} + C, \qquad a \neq 0 $$
  $$ \int\!\frac{1}{x}\dif x = \ln x + C, \qquad \qquad $$
  $$ \int\!\sec^2 x\dif x = \tan x + C. \qquad \qquad $$
}

\makesection{Técnicas de Integração}
\makesubsection{Substituição}

\myframe{
  A substituição serve para resolver problemas do tipo
  $$ \int\!f'(g(x))g'(x)\dif x. $$
  Criamos a variável $u = g(x)$. O diferencial de $u$ é $\dif u = g'(x)\dif x$.
  Substituindo na integral, temos
  $$ \int\!f'(u)\dif u $$
  Agora a integral é na variável $u$. Resolvemos normalmente:
  $$ \int\!f'(u)\dif u = f(u) + C = f(g(x)) + C$$
  A parte difícil de técnica é descobrir qual a substituição. Uma dica:
  em geral é a parte que mais complica (mas nem sempre).
}

\makesubsection{Exemplos}

\myframe{
  $$ \int\!e^{x^2}x\dif x $$
  A substituição aqui é $u = x^2$. Temos $\dif u = 2x\dif x$, de
  modo que $\dif x = \frac{1}{2x}\dif u$.
  Substituindo, temos
  $$ \int\!e^ux\frac{1}{2x}\dif u = \int\!e^u\frac{1}{2}\dif u 
    = \frac{1}{2}\int\!e^u\dif u = \frac{1}{2}e^u + C 
    = \frac{1}{2}e^{x^2} + C$$
}

\myframe{
  $$ \int\!\sqrt{x^3+1}x^2\dif x $$
  A substituição aqui é $u = x^3 + 1$. Temos
  $\dif u = 3x^2\dif x$, de modo que $\dif x = \frac{1}{3x^2}\dif u$.
  Substituindo, temos
  \begin{align*}
    \int\!\sqrt{u}x^2\frac{1}{3x^2}\dif u & =
    \int\!\sqrt{u}\frac{1}{3}\dif u =
    \frac{1}{3}\int\!u^{1/2}\dif u =
    \frac{1}{3}\frac{u^{3/2}}{3/2} + C \\ 
    & = \frac{1}{3}\frac{2}{3}(x^3+1)^{3/2} + C
    = \frac{2(x^3 + 1)^{3/2}}{9} + C
  \end{align*}
}

\makesubsection{Integral por Partes}

\myframe{
  A técnica de integral por partes serve para integrais
  do tipo
  $$ \int\!f'(x)g(x)\dif x $$
  Teremos duas variáveis $u$ e $v$, escolhidas
  de modo que
  $$ u = g(x) \qquad \mbox{e} 
    \qquad \dif v = f'(x)\dif x $$
  A mudança transforma a integral em
  $$ \int\!u\dif v $$
  Pela regra do produto para derivadas, temos
  $ \dif(uv) = u\dif v + v\dif u, $
  de modo que
  $$ \int\!u\dif v = uv - \int\!v\dif u $$
}

\myframe {
  $$ \int\!u\dif v = uv - \int\!v\dif u $$
  $$ \int\!f'(x)g(x)\dif x = f(x)g(x) 
    - \int\!f(x)g'(x)\dif x $$

  A dificuldade desta técnica é fazer a separação
  das funções. Normalmente a escolha envolve saber
  o que é melhor integrar e o que é melhor derivar.
}

\makesubsection{Exemplos}

\myframe{
  $$ \int\!xe^x\dif x $$
  Para esta integral, escolhemos
  $ u = x $ e $\dif v = e^x\dif x$. Dessa escolha,
  segue 
  $$ v = e^x \qquad \mbox{e} \qquad \dif u = \dif x. $$
  Então
  \begin{align*}
  \int\!xe^x\dif x & = \int\!u\dif v =
    uv - \int\!v\dif u = xe^x - \int\!e^x\dif x  \\
  & = xe^x - e^x + C
  \end{align*}
}

\myframe{
  $$ \int\!\ln x\dif x $$
  Para esta integral, escolhemos
  $ u = \ln x$ e $\dif v = \dif x$. Dessa escolha,
  segue
  $$ \dif u = \frac{1}{x}\dif x \qquad
    \mbox{e} \qquad v = x. $$
  Então
  \begin{align*}
    \int\!\ln x\dif x & = \int\!u\dif v =
      uv - \int\!v\dif u = x\ln x - \int\!\frac{1}{x}x\dif x \\
    & = x\ln x - \int\!\dif x = x\ln x - x + C
  \end{align*}
}

\end{document}

\documentclass[10 pt]{beamer}
\usepackage[brazil]{babel}
\usepackage[T1]{fontenc}
\usepackage{ae}
\usepackage[utf8]{inputenc}
\usepackage{framed}
%\usepackage[dvipsnames]{color}
\usepackage{graphicx}
\usepackage{algorithm}
\usepackage{algpseudocode}
\usepackage{epsfig}
\usepackage{tikz}
\usepackage{pgfplots}
\usetikzlibrary{plotmarks}
\usepackage{amssymb, amsmath, amsfonts}
\bibliographystyle{plain}
%\topmargin		0 cm
%\hoffset		0 cm
%\voffset		0 cm
%\evensidemargin		0 cm
%\oddsidemargin		0 cm
%\setlength{\textwidth}{16 cm}
%\setlength{\textheight}{21 cm}

\colorlet{mybasecolor}[rgb]{teal}

\colorlet{mydarkestcolor}[rgb]{mybasecolor!20!black}
\colorlet{mydarkcolor}[rgb]{mybasecolor!60!black}
\colorlet{mynormalcolor}[rgb]{mybasecolor!90!black}
\colorlet{mylightcolor}[rgb]{mybasecolor!10!white}

\colorlet{mysidebarcolor}[rgb]{mydarkcolor}
\colorlet{mylogobg}[rgb]{mylightcolor}
\colorlet{mylogofg}[rgb]{mydarkestcolor}
\colorlet{mysidebarauthor}[rgb]{mylightcolor}
\colorlet{mysidebartitle}[rgb]{mylightcolor}
\colorlet{mysidebarsecfg}[rgb]{mydarkestcolor}
\colorlet{mysidebarsecbg}[rgb]{mylightcolor}
\colorlet{mysidebarsubsecfg}[rgb]{mydarkestcolor}
\colorlet{mysidebarsubsecbg}[rgb]{mylightcolor}
\colorlet{myframetitlefg}[rgb]{mylightcolor}
\colorlet{myframetitlebg}[rgb]{mydarkcolor}
\colorlet{myitemcolor}[rgb]{mynormalcolor}
\colorlet{myfontcolor}[rgb]{mydarkestcolor}
\colorlet{mytitlefg}[rgb]{mylightcolor}
\colorlet{mytitlebg}[rgb]{mydarkcolor}
\colorlet{myauthorfg}[rgb]{mylightcolor}
\colorlet{myauthorbg}[rgb]{mydarkcolor}

\useinnertheme{rounded}
\setbeamercolor{structure}{fg = myitemcolor}
%\useoutertheme[height=1 cm,width=1.6 cm]{sidebar}
\useoutertheme{smoothtree}

\setbeamercolor{palette secondary}{bg=mylogobg,fg=mylogofg} %cor do logo
\setbeamercolor{sidebar}{bg=mysidebarcolor} %Background do sidebar
\setbeamercolor{palette sidebar tertiary}{fg = mysidebarauthor} %Autor no sidebar
\setbeamercolor{palette sidebar quaternary}{fg = mysidebartitle} %Titulo no sidebar
\setbeamercolor{section in sidebar}{fg = mysidebarsecfg,bg = mysidebarsecbg} %A cor eh a media de fg e bg
\setbeamercolor{subsection in sidebar}{fg = mysidebarsubsecfg,bg = mysidebarsubsecbg} %A cor eh a media de fg e bg
\setbeamercolor{frametitle}{fg = myframetitlefg, bg = myframetitlebg} 
\setbeamercolor{title}{bg = mytitlebg,fg = mytitlefg}
\setbeamerfont{title}{series = \bf}
\setbeamercolor{author}{bg = myauthorbg,fg = myauthorfg}
\setbeamerfont{author}{series = \it}
\setbeamercolor{normal text}{bg = white, fg = myfontcolor}
%\setbeamercolor{normal text}{bg = black, fg = white}
%\setbeamerfont{normal text}{family=serif}
\usefonttheme[onlymath]{serif}
%\logo{\includegraphics[scale=0.35]{logo.png}}
%\logo{$\min f(x)$}

\newcommand{\makesection}[1]{\section[#1]{#1}}
\newcommand{\makesubsection}[1]{\subsection[#1]{#1}}
\newcommand{\tikzcircle}{arc (0:360:0.01)}

\algrenewcommand\algorithmicif{\textbf{se}}
\algrenewcommand\algorithmicthen{\textbf{então}}
\algrenewcommand\algorithmicend{\textbf{fim do}}
\algrenewcommand\algorithmicwhile{\textbf{enquanto}}
\algrenewcommand\algorithmicdo{\textbf{faça}}
\algrenewcommand\algorithmicelse{\textbf{senão}}

\title{ MS380 - Aula 4 \\
Limites}
\author{Abel Soares Siqueira \\
Laércio Luís Vendite}
\date{}

\newcommand{\myframe}[1]{
\begin{frame}
 \frametitle{\insertsection \qquad {\small \insertsubsection}}
#1
\end{frame}}
\newcommand{\myframetop}[1]{
\begin{frame}[t]
 \frametitle{\insertsection \qquad {\small \insertsubsection}}
#1
\end{frame}}
% \newcommand{\visiblelbl}[1]{\label{#1}{\color{red}{#1}}}
\newcommand{\visiblelbl}[1]{\label{#1}}
\newcommand{\spc}{\vspace{0.5 cm}}
\newcommand{\systemtwo}[1]{$
  \left\{\begin{array}{rcrcr} #1
  \end{array}\right.$}
\newcommand{\systemthree}[1]{$
  \left\{\begin{array}{rcrcrcr}
  #1
  \end{array}\right.$}

\newcommand{\limninf}{\lim_{n\rightarrow\infty}}
\newcommand{\limx}[1]{\lim_{x\rightarrow{#1}}}
\newcommand{\fundef}[1]{\left\{\begin{array}{ll} {#1} \end{array}\right.}

\begin{document}

\begin{frame}
 \titlepage
\end{frame}

\makesection{Sequências}

\myframe {
  Uma sequência é uma lista ordenada de elementos.
  $$ (a_1, a_2, a_3, a_4, \dots) $$
  \pause
  Exemplos:
  $$ (1,1,1,\dots) \qquad \mbox{Sequência constante} $$
  \pause
  $$ (1,2,4,8,16,\dots) \qquad \mbox{PG com razão } 2$$
  \pause
  $$ \bigg(1, \frac{1}{2}, \frac{1}{2^2}, \frac{1}{2^3}, \frac{1}{2^4}, 
      \frac{1}{2^5},\dots\bigg) \qquad \mbox{PG com razão }\frac{1}{2}$$
  \pause
  $$ (2,5,8,11,14,\dots) \qquad \mbox{PA com razão } 3 $$
}

\myframe {
  Podemos escrever $S = (a_n)$ e descrever $a_n$.

  Exemplos:
  \begin{enumerate}
    \item<1-> $a_n = 2^n, \Rightarrow S = (2,4,8,16,\dots)$
    \item<2-> $a_n = \frac{1}{n}, 
      \Rightarrow S = (1, \frac{1}{2}, \frac{1}{3}, \frac{1}{4}, \dots) $
    \item<3-> $a_n = n^2, \Rightarrow S = (1, 2, 4, 9, 16, \dots)$
    \item<4-> $a_1 = 1$, $a_{n+1} = 2a_n + 1, \Rightarrow S = (1,3,7,15,31,63,\dots)$
    \item<5-> $a_1 = 2$, $a_{n+1} = \frac{1}{a_n} + 1, \Rightarrow
      S = (2,\frac{3}{2}, \frac{5}{3}, \frac{8}{5}, \frac{13}{8}, \dots)$
    \item<6-> $a_1 = 1, a_2 = 1$, $a_{n+1} = a_n + a_{n-1}, \Rightarrow
      S = (1, 1, 2, 3, 5, 8, 13, 21, \dots) $
  \end{enumerate}
}

\myframe {
  Conforme $n$ aumenta, a sequência $a_n$ pode se aproximar de algum valor $L$.

  Nesse caso, dizemos que $n$ converge para $L$ e dizemos que o limite da
  sequência $a_n$ para $n$ tendendo ao infinito é $L$. Escrevemos
  $$ \lim_{n\rightarrow\infty} a_n = L. $$

  Caso contrário, a sequência diverge, podendo tender ao infinito, ou
  não tender a nada.
}

\myframe {
  $$ a_n = \frac{1}{n} $$
  
  \begin{center}
  \begin{tikzpicture}[x=4cm,y=4cm]
    %\draw[very thin, color=gray] (-0.1,-0.1) grid (2.1, 1.1);
    \draw[->] (0,-0.1) -- (0,1.1);
    \draw[->] (-0.1,0) -- (2.1,0);
    \foreach \x in {1,2,...,20} {
      \draw (\x/10+0.01,1/\x) \tikzcircle {};
      \draw (\x/10,0.01) -- (\x/10,-0.01);
    }
  \end{tikzpicture}
  \end{center}
  Nesse caso, a sequência converge para 0.
}

\myframe {
  $$ a_n = \frac{n}{n+1} $$

  \begin{center}
  \begin{tikzpicture}[x=4cm,y=4cm]
    \draw[->] (0,-0.1) -- (0,1.1);
    \draw[->] (-0.1,0) -- (2.1,0);
    \foreach \x in {1,2,...,20} {
      \draw (\x/10+0.01,{\x/(\x+1)}) \tikzcircle {};
      \draw (\x/10,0.01) -- (\x/10,-0.01);
    }
    \draw[dashed] (-0.1,1.0) node[left] {$1$} -- (2.1,1.0);
  \end{tikzpicture}
  \end{center}
  Nesse caso, a sequência converge para 1.
}

\myframe {
  $$ a_n = n $$

  \begin{center}
  \begin{tikzpicture}[x=4cm,y=4cm]
    \draw[->] (0,-0.1) -- (0,1.1);
    \draw[->] (-0.1,0) -- (2.1,0);
    \foreach \x in {1,2,...,20} {
      \draw (\x/10+0.01,\x/20) \tikzcircle {};
      \draw (\x/10,0.01) -- (\x/10,-0.01);
      \draw (-0.01,\x/20) -- (0.01,\x/20);
    }
    \foreach \x in {5,10,15,20} {
      \draw[thin,gray,dashed] (0.0,\x/20) node[left] {$\x$} -- (2.1,\x/20);
      \draw[thin,gray,dashed] (\x/10,0.0) node[below] {$\x$} -- (\x/10,1.1);
    }
  \end{tikzpicture}
  \end{center}
  Nesse caso, a sequência tende a infinito, logo diverge.
}

\myframe {
  $$ a_n = (-1)^n $$

  \begin{center}
  \begin{tikzpicture}[x=4cm,y=4cm]
    \draw[->] (0,-0.6) -- (0,0.6);
    \draw[->] (-0.1,0) -- (2.1,0);
    \foreach \x in {1,2,...,20} {
      \draw (\x/10+0.01,{0.5*(-1)^\x}) \tikzcircle {};
      \draw (\x/10,0.01) -- (\x/10,-0.01);
    }
    \draw[thin,gray,dashed] (-0.01,0.5) node[left] {$1$} -- (2.1,0.5);
    \draw[thin,gray,dashed] (-0.01,-0.5) node[left] {$-1$} -- (2.1,-0.5);
  \end{tikzpicture}
  \end{center}
  Nesse caso, a sequência não tende a nada, alternando sempre entre -1 e 1. 
    Portanto, é divergente.
}

\section{Definição de Limite de Sequências}

\myframe {
  Dado uma sequência real $(a_n)$, se existir $L \in \R$ tal que
  para todo $\varepsilon > 0$, existe $n_0 \in \mathbb{N}$ tal que
  $\modulo{a_n - L} < \varepsilon$ para todo $n > n_0$.

  Escrevemos $$\lim_{n\rightarrow\infty} a_n = L.$$

  \begin{center}
  \begin{tikzpicture}
    \draw[->] (-0.2,0) -- (4.1,0);
    \draw[fill=red] (4,0) circle (0.02) node[below] {$a_1$};
    \draw[fill=red] (4/2,0) circle (0.02) node[below] {$a_2$};
    \draw[fill=red] (4/4,0) circle (0.02) node[below] {$a_3$};
    \draw[fill=red] (4/8,0) circle (0.02) node[below] {$a_4$};
    \draw[fill=red] (4/16,0) circle (0.02);
    \draw[fill=red] (4/32,0) circle (0.02);
    \draw[fill=red] (4/64,0) circle (0.02);
    \draw[fill=red] (4/128,0) circle (0.02);
    \draw (0,0.05) -- (0,-0.05) node[below] {$L$};
  \end{tikzpicture}
  \end{center}
}

\subsection{Exemplos}

\myframe{
  \begin{itemize}
    \item $ \displaystyle \lim_{n\rightarrow\infty}\frac{1}{n} = 0, $
    \item $ \displaystyle \lim_{n\rightarrow\infty}\frac{n+1}{n} = 1, $
    \item $ \displaystyle \lim_{n\rightarrow\infty}\frac{n^2-5n+6}{4n^2 + 1} = \frac{1}{4}, $
    \item $ \displaystyle \lim_{n\rightarrow\infty}n = +\infty, $
    \item $ \displaystyle \lim_{n\rightarrow\infty}(-1)^n \mbox{ n\~ao existe}, $
    \item $ \displaystyle \lim_{n\rightarrow\infty}\sqrt{n+a^2} - \sqrt{n} = 0, $
    \item $ \displaystyle \lim_{n\rightarrow\infty}ne^{-n} = 0, $
  \end{itemize}
}

\subsection{Operações}

\myframe{
  $$ \limninf a_n = L_1 \qquad \mbox{ e } \qquad \limninf b_n = L_2 $$
  \begin{align*}
    \limninf a_n\pm b_n & = \limninf a_n \pm \limninf b_n = L_1\pm L_2 \\
    \limninf a_nb_n & = \limninf a_n\limninf b_n = L_1L_2 \\
    \limninf \frac{a_n}{b_n} & = \frac{\limninf a_n}{\limninf b_n}
      = \frac{L_1}{L_2}, \qquad \mbox{se } L_2 \neq 0 \\
    \limninf a_n^p & = \bigg(\limninf a_n\bigg)^p \\
    \limninf \sqrt{a_n} & = \sqrt{\limninf a_n}
  \end{align*}
}

\section{Limites Fundamentais}

\subsection{Progressão Geométrica}

\myframe {
  $$a_n = a_0q^n$$

  $$ \limninf a_n = \left\{
    \begin{array}{ll}
      0, & \modulo{q} < 1 \\
      a_0, & q = 1 \\
      \infty, & q > 1 \\
      \mbox{não existe}, & q \leq -1
    \end{array}
    \right. $$
}

\subsection{Frações polinomiais}

\myframe {
  $$ a_n = \frac{p(n)}{q(n)} $$

  $$ p(n) = p_0n^k + p_1n^{k-1} + \dots + p_k, \qquad \mbox{grau}(p) = k $$
  $$ q(n) = q_0n^m + q_1n^{m-1} + \dots + q_m, \qquad \mbox{grau}(q) = m $$
  
  $$ \limninf a_n = \left\{
    \begin{array}{ll}
      0, & k < m \\ 
      \frac{p_0}{q_0}, & k = m \\
      \mbox{sinal}(p_0*q_0)\infty, & k > m \\
    \end{array}
    \right. $$
}

\myframe {
  \begin{align*}
    \limninf \frac{2n + 1}{3n - 1} & =
      \limninf \frac{n(2 + \frac{1}{n})}{n(3 - \frac{1}{n})} =
      \limninf \frac{2 + \frac{1}{n}}{3 - \frac{1}{n}} = \frac{2}{3}
  \end{align*}
  \vspace{0.5 cm}
  \begin{align*}
    \limninf \frac{2n - 1}{2n^2 + 3n - 1} & =
      \limninf \frac{n^2(\frac{2}{n} - \frac{1}{n^2})}{n^2(2 + \frac{3}{n} - \frac{1}{n})} =
      \limninf \frac{\frac{2}{n} - \frac{1}{n^2}}{2+\frac{3}{n}-\frac{1}{n}} = \frac{0}{2} = 0
  \end{align*}
  \vspace{0.5 cm}
  \begin{align*}
    \limninf \frac{-n^2 + 2n}{3n + 1} & =
      \limninf \frac{n^2(-1 + \frac{2}{n})}{n^2(\frac{3}{n}+\frac{1}{n^2})} =
      \limninf \frac{-1+\frac{2}{n}}{\frac{3}{n}+\frac{1}{n^2}}
  \end{align*}
}

\myframe {
  \begin{align*}
    \limninf \bigg(1 + \frac{1}{n}\bigg)^n = e \approx 2.71828
  \end{align*}
  \vspace{0.5 cm}
  \begin{align*}
    \limninf \bigg(1 + \frac{a}{n}\bigg)^n = e^a
  \end{align*}
}

\section{Limite de Função}

\myframe {
  Dado uma função $f$, com $a \in \R$, se existe $L \in \R$
  tal que, para todo $\varepsilon > 0$, existe $\delta > 0$ (que depende
  de $\varepsilon$) tal que
  $$\mbox{se } \qquad 0 < \modulo{x - a} < \delta, \qquad \mbox{então} \qquad
      \modulo{f(x) - L} < \varepsilon$$
  \begin{center}
    \begin{tikzpicture}[domain=0.0:3.5]
      \draw[->] (-0.1,0) -- (4.1,0);
      \draw[->] (0,-0.1) -- (0,4.1);
      \draw plot (\x,0.25*\x^2+1);
      \draw[dashed,gray] (2,4.1) -- (2,-0.1) node[below] {$a$};
      \draw[dashed,gray] (4.1,2) -- (-0.1,2) node[left] {$L$};
      \only<1>{
        \draw[<->] (4,2-0.9) -- node[right] {$\varepsilon$} (4,2);
        \draw[<->] (2-0.6,4) -- node[above] {$\delta$} (2,4);
        \draw[color=red] (0,2-0.9) -- (4.1,2-0.9);
        \draw[color=red] (0,2+0.9) -- (4.1,2+0.9);
        \draw[color=blue] (2-0.6,0) -- (2-0.6,4.1);
        \draw[color=blue] (2+0.6,0) -- (2+0.6,4.1);
        \draw[color=blue,domain=1.4:2.6,thick] plot (\x,0.25*\x^2+1);
      }
      \only<2>{
        \draw[<->] (4,2-0.7) -- node[right] {$\varepsilon$} (4,2);
        \draw[<->] (2-0.4,4) -- node[above] {$\delta$} (2,4);
        \draw[color=red] (0,2-0.7) -- (4.1,2-0.7);
        \draw[color=red] (0,2+0.7) -- (4.1,2+0.7);
        \draw[color=blue] (2-0.4,0) -- (2-0.4,4.1);
        \draw[color=blue] (2+0.4,0) -- (2+0.4,4.1);
        \draw[color=blue,domain=2-0.4:2+0.4,thick] plot (\x,0.25*\x^2+1);
      }
      \only<3>{
        \draw[<->] (4,2-0.4) -- node[right] {$\varepsilon$} (4,2);
        \draw[<->] (2-0.2,4) -- node[above] {$\delta$} (2,4);
        \draw[color=red] (0,2-0.4) -- (4.1,2-0.4);
        \draw[color=red] (0,2+0.4) -- (4.1,2+0.4);
        \draw[color=blue] (2-0.2,0) -- (2-0.2,4.1);
        \draw[color=blue] (2+0.2,0) -- (2+0.2,4.1);
        \draw[color=blue,domain=2-0.2:2+0.2,thick] plot (\x,0.25*\x^2+1);
      }
      \only<4>{
        \draw[<->] (4,2-0.2) -- node[right] {$\varepsilon$} (4,2);
        \draw[<->] (2-0.1,4) -- node[above] {$\delta$} (2,4);
        \draw[color=red] (0,2-0.2) -- (4.1,2-0.2);
        \draw[color=red] (0,2+0.2) -- (4.1,2+0.2);
        \draw[color=blue] (2-0.1,0) -- (2-0.1,4.1);
        \draw[color=blue] (2+0.1,0) -- (2+0.1,4.1);
        \draw[color=blue,domain=2-0.1:2+0.1,thick] plot (\x,0.25*\x^2+1);
      }
    \end{tikzpicture}
  \end{center}
}

\myframe {
  $$\lim_{x\rightarrow3} 4x+2 = 14$$
  $$\lim_{x\rightarrow2} 3x^2+1 = 13$$
  $$\lim_{x\rightarrow6} \frac{3}{x+3} = \frac{1}{3}$$
  $$\lim_{x\rightarrow2} \frac{x^3-4x}{x^2-4} = ? 
    {\color{red} \bigg(\frac{0}{0}\bigg)}$$
  $$\lim_{x\rightarrow5} \frac{x-5}{x^2-5x} = ?
    {\color{red} \bigg(\frac{0}{0}\bigg)}$$
}

\myframe {
  A função não precisa estar definida no ponto para ter limite naquele ponto.
  E mesmo que esteja definida no ponto, o limite não precisa ser igual ao valor
  da função.
  $$ f(x) = \left\{\begin{array}{ll}
    x, & x \neq 2 \\
    0, & x = 2
  \end{array}\right.$$
  \begin{center}
  \begin{tikzpicture}[scale=0.8]
    \draw[->] (-0.1,0) -- (3.6,0);
    \draw[->] (0,-0.1) -- (0,3.6);
    \draw[domain=0.5:1.97] plot (\x,\x);
    \draw[domain=2.03:3.5] plot (\x,\x);
    \draw (2,2) circle (0.04);
    \draw[fill] (2,0) circle (0.04);
    \foreach \x in {1,2,3} {
      \draw (-0.1,\x) node[left] {\x} -- (0.1,\x);
      \draw (\x,-0.1) node[below] {\x} -- (\x,0.1);
    }
  \end{tikzpicture}
  \end{center}
  $$ \limx{2} f(x) = 2, \qquad \mbox{mas} \qquad f(2) = 0$$
}

\subsection{Indeterminação}

\myframe {
  \begin{align*}
    \lim_{x\rightarrow2} \frac{x^3-4x}{x^2-4} & =
      \lim_{x\rightarrow2} \frac{x(x^2 - 4)}{x^2-4} =
      \lim_{x\rightarrow2} x = 2
  \end{align*}
  \begin{align*}
    \limx{5} \frac{x-5}{x^2-5x} & = 
      \limx{5} \frac{x-5}{x(x-5)} =
      \limx{5} \frac{1}{x} = \frac{1}{5}
  \end{align*}
  \begin{align*}
    \limx{2} \frac{\sqrt{x}-\sqrt{2}}{x-2} & =
      \limx{2} \frac{(\sqrt{x}-\sqrt{2})(\sqrt{x}+\sqrt{2})}{(x-2)(\sqrt{x}+\sqrt{2})} =
      \limx{2} \frac{x-2}{(x-2)(\sqrt{x}+\sqrt{2})} \\
      & = \limx{2} \frac{1}{\sqrt{x}+\sqrt{2}} = \frac{1}{2\sqrt{2}}
  \end{align*}
}

\subsection{Limites Laterais}

\myframe {
  $$ \limx{a^+}f(x) \qquad \mbox{limite com $x$ tendendo a $a$ por valores maiores que $a$}$$
  $$ \limx{a^-}f(x) \qquad \mbox{limite com $x$ tendendo a $a$ por valores menores que $a$}$$
  \vspace{0.5 cm}
  \begin{center}
    $\displaystyle \limx{a}f(x)$ só existe se os limites laterais são iguais, e o
    limite é igual aos limites laterais.
  \end{center}
}

\myframe {
  $$ f(x) = \left\{\begin{array}{ll}
    x, & x \leq 1 \\
    3-x, & x > 1
  \end{array}\right. $$
  \begin{center}
  \begin{tikzpicture}
    \draw[->] (-0.1,0) -- (2.1,0);
    \draw[->] (0,-0.1) -- (0,2.1);
    \draw[domain=-0.1:1,blue] plot (\x,\x);
    \draw[domain=1:2.1,blue] plot (\x,3-\x);
    \draw[fill] (1,1) circle (0.03);
    \draw (1,2) circle (0.03);
    \draw[dashed] (1,0) node[below] {1} -- (1,2);
    \foreach \x in {1,2} {
      \draw (0.1,\x) -- (-0.1,\x) node[left] {\x};
    }
  \end{tikzpicture}
  \end{center}
  $$ \limx{1^+} f(x) = \limx{1^+} 3-x = 2 \qquad \limx{1^-} f(x) = \limx{1^-} x = 1$$
  $$ \limx{1} f(x) \mbox{ não existe} $$
}

\subsection{Limites Infinitos}

\myframe {
  $$ f(x) = \frac{1}{x} $$
  \begin{center}
  \begin{tikzpicture}[scale=0.5]
    \draw[->] (-4.1,0) -- (4.1,0);
    \draw[->] (0,-4.1) -- (0,4.1);
    \draw[blue,domain=-4:-1/4] plot (\x,1/\x);
    \draw[blue,domain=1/4:4] plot (\x,1/\x);
  \end{tikzpicture}
  \end{center}
  $$ \limx{0^+} \frac{1}{x} = +\infty \qquad \qquad
    \limx{0^-} \frac{1}{x} = -\infty $$
  $$ \limx{+\infty} \frac{1}{x} = 0 \qquad \qquad
    \limx{-\infty} \frac{1}{x} = 0 $$
}

\myframe {
  $$ f(x) = \frac{(x-1)^2}{x} $$
  \begin{center}
  \begin{tikzpicture}[scale=0.25]
    \draw[->] (-4.1,0) -- (4.1,0);
    \draw[->] (0,-8.1) -- (0,4.1);
    \draw[blue,domain=1/6:4] plot (\x,{(\x-1)^2/\x});
    \draw[blue,domain=-4:-1/6] plot (\x,{(\x-1)^2/\x});
  \end{tikzpicture}
  \end{center}
  $$ \limx{0^+} \frac{(x-1)^2}{x} = +\infty \qquad \qquad
    \limx{0^-} \frac{(x-1)^2}{x} = -\infty $$
  $$ \limx{+\infty} \frac{(x-1)^2}{x} = +\infty \qquad \qquad
    \limx{-\infty} \frac{(x-1)^2}{x} = -\infty $$
}

\subsection{Operações}

\myframe {
  $$ \limx{a} f(x) = F \qquad \limx{a} g(x) = G $$
  \begin{align*}
    \limx{a} f(x)\pm g(x) & = \limx{a} f(x) \pm \limx{a} g(x) = F\pm G \\
    \limx{a} f(x)g(x) & = \limx{a} f(x) \limx{a} g(x) = FG \\
    \limx{a} \dfrac{f(x)}{g(x)} & = 
      \dfrac{\displaystyle\limx{a} f(x)}{\displaystyle\limx{a} g(x)} = 
      \frac{F}{G}, \qquad \mbox{se } G \neq 0 \\
    \limx{a} f(x)^n & = \bigg[\limx{a} f(x)\bigg]^n
  \end{align*}
}

\subsection{Teorema do Confronto}

\myframe {
  Se $f(x) \leq g(x) \leq h(x)$ para todo $x$ numa vizinhança de $a$,
    exceto, possivelmente em $a$, e
    $$ \limx{a} f(x) = L = \limx{a} h(x) = L, $$
    então
    $$ \limx{a} g(x) = L $$
  \begin{center}
  \begin{tikzpicture}[domain=-2:2,scale=1]
    \draw[blue] plot(\x,0.3*\x^2);
    \draw[blue] plot(\x,-0.4*\x^2);
    \draw[red] plot(\x,-0.2*\x^3);
    \draw[dashed] (-2,0) node[left] {$L$} -- 
      (0,0) -- (0,-2) node[below] {$a$};
  \end{tikzpicture}
  \end{center}
}

\myframe {
  Calcule o limite $\displaystyle \limx{0}x^2\sin\frac{1}{x}$.
  \vspace{0.5cm}

  Como $-1 \leq \sin\frac{1}{x} \leq 1$, então
  $$ -x^2 \leq x^2\sin\frac{1}{x} \leq x^2. $$
  E
  $$ \limx{0} -x^2 = \limx{0} x^2 = 0. $$
  Logo,
  $$ \limx{0} x^2\sin\frac{1}{x} = 0. $$
}


\subsection{Limite do Seno}

\myframe {
  $$ \limx{0} \frac{\sin x}{x} = 1 $$
  \vspace{0.5 cm}

  Basta ver que
  $$ \sin x < x < \tan x $$
  $$ 1 < \frac{x}{\sin x} < \frac{1}{\cos x} $$
  $$ \cos x < \frac{\sin x}{x} < 1 $$

  \begin{center}
  \begin{tikzpicture}
    \draw[->] (-3.1,0) -- (3.1,0);
    \draw[->] (0,-0.1) -- (0,1.3);
    \draw[domain=-3:-0.02,blue] plot (\x,{sin(\x r)/\x});
    \draw[domain=0.02:3,blue] plot (\x,{sin(\x r)/\x});
    \draw (0,1) circle (0.03);
    \draw[dashed] (-3,1) -- node[above right] {$1$} (3,1);
  \end{tikzpicture}
  \end{center}
}

\section{Funções Contínuas}

\myframe {
  Seja $f$ definida no intervalo aberto $(a,b)$.
  $f$ é {\bf contínua em $c \in (a,b)$} se o limite de $f(x)$ para
  $x$ tendendo a $c$ existe e é igual a $f(c)$, isto é, se
  $$ \limx{c} f(x) = f(c). $$

  Uma função é dita contínua se é contínua em todos os pontos de seu domínio.
  
  \begin{itemize}
    \item Todos os polinômios são funções contínuas.
    \item $f(x) = e^x$, $f(x) = \sin(x)$, $f(x) = \cos(x)$, são funções contínuas.
    
    \item $f(x) = 
    \left\{
    \begin{array}{ll} 
      0, & x \leq 0 \\ 
      1, & x > 0
    \end{array}
    \right.$ 
      não é contínua no $0$, mas
      é contínua em todos os outros pontos.
    \item $f(x) = 
    \left\{
    \begin{array}{ll} 
      \dfrac{\sin x}{x}, & x \neq 0 \\ 
      1, & x = 0
    \end{array}
    \right.$ 
    é contínua em todos os pontos.
    \item $f(x) = 
    \left\{
    \begin{array}{ll} 
      1, & x \in \mathbb{Q} \\ 
      0, & x \not\in \mathbb{Q}
    \end{array}
    \right.$ 
    não é contínua em nenhum ponto.
  \end{itemize}
}

\end{document}

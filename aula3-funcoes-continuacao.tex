\documentclass[10 pt]{beamer}
\usepackage[brazil]{babel}
\usepackage[T1]{fontenc}
\usepackage{ae}
\usepackage[utf8]{inputenc}
\usepackage{framed}
%\usepackage[dvipsnames]{color}
\usepackage{graphicx}
\usepackage{algorithm}
\usepackage{algpseudocode}
\usepackage{epsfig}
\usepackage{tikz}
\usepackage{pgfplots}
\usetikzlibrary{fit}
\usepackage{amssymb, amsmath, amsfonts}
\bibliographystyle{plain}
%\topmargin		0 cm
%\hoffset		0 cm
%\voffset		0 cm
%\evensidemargin		0 cm
%\oddsidemargin		0 cm
%\setlength{\textwidth}{16 cm}
%\setlength{\textheight}{21 cm}

\colorlet{mybasecolor}[rgb]{teal}

\colorlet{mydarkestcolor}[rgb]{mybasecolor!20!black}
\colorlet{mydarkcolor}[rgb]{mybasecolor!60!black}
\colorlet{mynormalcolor}[rgb]{mybasecolor!90!black}
\colorlet{mylightcolor}[rgb]{mybasecolor!10!white}

\colorlet{mysidebarcolor}[rgb]{mydarkcolor}
\colorlet{mylogobg}[rgb]{mylightcolor}
\colorlet{mylogofg}[rgb]{mydarkestcolor}
\colorlet{mysidebarauthor}[rgb]{mylightcolor}
\colorlet{mysidebartitle}[rgb]{mylightcolor}
\colorlet{mysidebarsecfg}[rgb]{mydarkestcolor}
\colorlet{mysidebarsecbg}[rgb]{mylightcolor}
\colorlet{mysidebarsubsecfg}[rgb]{mydarkestcolor}
\colorlet{mysidebarsubsecbg}[rgb]{mylightcolor}
\colorlet{myframetitlefg}[rgb]{mylightcolor}
\colorlet{myframetitlebg}[rgb]{mydarkcolor}
\colorlet{myitemcolor}[rgb]{mynormalcolor}
\colorlet{myfontcolor}[rgb]{mydarkestcolor}
\colorlet{mytitlefg}[rgb]{mylightcolor}
\colorlet{mytitlebg}[rgb]{mydarkcolor}
\colorlet{myauthorfg}[rgb]{mylightcolor}
\colorlet{myauthorbg}[rgb]{mydarkcolor}

\useinnertheme{rounded}
\setbeamercolor{structure}{fg = myitemcolor}
%\useoutertheme[height=1 cm,width=1.6 cm]{sidebar}
\useoutertheme{smoothtree}

\setbeamercolor{palette secondary}{bg=mylogobg,fg=mylogofg} %cor do logo
\setbeamercolor{sidebar}{bg=mysidebarcolor} %Background do sidebar
\setbeamercolor{palette sidebar tertiary}{fg = mysidebarauthor} %Autor no sidebar
\setbeamercolor{palette sidebar quaternary}{fg = mysidebartitle} %Titulo no sidebar
\setbeamercolor{section in sidebar}{fg = mysidebarsecfg,bg = mysidebarsecbg} %A cor eh a media de fg e bg
\setbeamercolor{subsection in sidebar}{fg = mysidebarsubsecfg,bg = mysidebarsubsecbg} %A cor eh a media de fg e bg
\setbeamercolor{frametitle}{fg = myframetitlefg, bg = myframetitlebg} 
\setbeamercolor{title}{bg = mytitlebg,fg = mytitlefg}
\setbeamerfont{title}{series = \bf}
\setbeamercolor{author}{bg = myauthorbg,fg = myauthorfg}
\setbeamerfont{author}{series = \it}
\setbeamercolor{normal text}{bg = white, fg = myfontcolor}
%\setbeamercolor{normal text}{bg = black, fg = white}
%\setbeamerfont{normal text}{family=serif}
\usefonttheme[onlymath]{serif}
%\logo{\includegraphics[scale=0.35]{logo.png}}
%\logo{$\min f(x)$}

\newcommand{\makesection}[1]{\section[#1]{#1}}
\newcommand{\makesubsection}[1]{\subsection[#1]{#1}}

\algrenewcommand\algorithmicif{\textbf{se}}
\algrenewcommand\algorithmicthen{\textbf{então}}
\algrenewcommand\algorithmicend{\textbf{fim do}}
\algrenewcommand\algorithmicwhile{\textbf{enquanto}}
\algrenewcommand\algorithmicdo{\textbf{faça}}
\algrenewcommand\algorithmicelse{\textbf{senão}}

\title{ MS380 - Aula 3 \\
Funções (continuação)}
\author{Abel Soares Siqueira \\
Laércio Luís Vendite}
\date{}

\newcommand{\myframe}[1]{
\begin{frame}
 \frametitle{\insertsection \qquad {\small \insertsubsection}}
#1
\end{frame}}
\newcommand{\myframetop}[1]{
\begin{frame}[t]
 \frametitle{\insertsection \qquad {\small \insertsubsection}}
#1
\end{frame}}
% \newcommand{\visiblelbl}[1]{\label{#1}{\color{red}{#1}}}
\newcommand{\visiblelbl}[1]{\label{#1}}
\newcommand{\spc}{\vspace{0.5 cm}}
\newcommand{\systemtwo}[1]{$
  \left\{\begin{array}{rcrcr} #1
  \end{array}\right.$}
\newcommand{\systemthree}[1]{$
  \left\{\begin{array}{rcrcrcr}
  #1
  \end{array}\right.$}

\begin{document}

\begin{frame}
 \titlepage
\end{frame}

\newcommand{\boxat}[4]{
  \draw[color=gray] (#1-0.5,#2-0.5) -- (#1+#3+0.5,#2-0.5) -- 
    (#1+#3+0.5,#2+#4+0.5) -- (#1-0.5,#2+#4+0.5) -- (#1-0.5,#2-0.5);}
\newcommand{\seta}[2]{ \draw[->] #1 -- #2}
\newcommand{\setaemph}[2]{ \draw[thick,color=red,->] #1 -- #2}
\makesection{Funções}

\makesubsection{Injetividade e Sobrejetividade}

\myframe {
  Uma função $f:A\rightarrow B$ é {\bf injetiva}, ou {\bf injetora}, se
  para qualquer $x,y\in A$, com $x \neq y$, 
  temos $f(x) \neq f(y)$.

  Isto é análogo a dizer que se $f(x) = f(y)$,
  então $x = y$.

  Significa que cada elemento do contra-domínio tem no máximo
  um elemento do domínio relacionado à ele.

  Exemplo:
  \begin{itemize}
    \item $f:\mathbb{Z}\rightarrow\mathbb{Z}, f(n) = n + 1$ é injetiva.
    \item $f:\mathbb{R}\rightarrow\mathbb{R}, f(x) = x^2$ não é injetiva.
    \item $f:[0,\infty)\rightarrow\mathbb{R}, f(x) = x^2$ é injetiva.
    \item $f:\mathbb{R}\rightarrow\mathbb{R}, f(x) = 10x - 4$ é injetiva.
    \item $f:\mathbb{R}\rightarrow\mathbb{R}, f(x) = e^x$ é injetiva.
    \item $f:\mathbb{R}^*\rightarrow\mathbb{R}, f(x) = \frac{1}{x}$ é injetiva.
    \item $f:\mathbb{R}\rightarrow\mathbb{R}, f(x) = \sin(x)$ não é injetiva.
    \item $f:[0,\pi/2]\rightarrow\mathbb{R}, f(x) = \sin(x)$ é injetiva.
  \end{itemize}
}

\myframe {
  Uma função $f:A\rightarrow B$ é {\bf sobrejetiva}, ou {\bf sobrejetora}, se
  para qualquer $y \in B$, existe $x \in A$
  tal que $f(x) = y$.

  Isto é análogo a dizer que a imagem de $f$ é igual a seu
  contra-domínio.

  Significa que todo elemento do contra-domínio tem ao menos
  um elemento do domínio relacionado à ele.

  Exemplo:
  \begin{itemize}
    \item $f:\mathbb{Z}\rightarrow\mathbb{Z}, f(n) = n + 1$ é sobrejetiva.
    \item $f:\mathbb{R}\rightarrow\mathbb{R}, f(x) = x^2$ não é sobrejetiva.
    \item $f:\mathbb{R}\rightarrow[0,\infty), f(x) = x^2$ é sobrejetiva.
    \item $f:\mathbb{R}\rightarrow[0,\infty), f(x) = e^x$ é sobrejetiva.
    \item $f:\mathbb{R}^*\rightarrow\mathbb{R}, f(x) = \frac{1}{x}$ não é sobrejetiva.
    \item $f:\mathbb{R}^*\rightarrow\mathbb{R}^*, f(x) = \frac{1}{x}$ é sobrejetiva.
    \item $f:\mathbb{R}\rightarrow\mathbb{R}, f(x) = \sin(x)$ não é sobrejetiva.
    \item $f:\mathbb{R}\rightarrow[-1,1], f(x) = \sin(x)$ é sobrejetiva.
  \end{itemize}
}

\myframe {
  Uma função é {\bf bijetiva}, ou {\bf bijetora}, se é ao mesmo tempo
  bijetiva e sobrejetiva.

  Exemplo:
  \begin{itemize}
    \item $f:\mathbb{Z}\rightarrow\mathbb{Z}, f(n) = n + 1$ é bijetora.
    \item $f:[0,\infty)\rightarrow[0,\infty), f(x) = x^2$ é bijetora.
    \item $f:\mathbb{R}\rightarrow(0,\infty), f(x) = e^x$ é bijetora.
    \item $f:(0,\infty)\rightarrow\mathbb{R}, f(x) = \ln x$ é bijetora.
    \item $f:\mathbb{R}^*\rightarrow\mathbb{R}^*, f(x) = \frac{1}{x}$ é bijetora.
    \item $f:[0,\infty)\rightarrow[0,\infty), f(x) = \sqrt{x}$ é bijetora.
  \end{itemize}
}

\makesubsection{Funções Compostas}

\myframe {
  \begin{center}
  \begin{tikzpicture}
    \node (f) at (1,5) {$f$};
    \node[draw,circle] (1) at (0,4){1};
    \node[draw,circle] (2) at (0,3){2};
    \node[draw,circle] (3) at (0,2){3};
    \node[draw,circle] (4) at (0,1){4};
    \node[draw,circle] (5) at (0,0){5};
    \node[draw,circle] (a) at (2,3){a};
    \node[draw,circle] (b) at (2,2){b};
    \node[draw,circle] (c) at (2,1){c};
    \node (g) at (7,5) {$g$};
    \node[draw,circle] (a2) at (6,3){a};
    \node[draw,circle] (b2) at (6,2){b};
    \node[draw,circle] (c2) at (6,1){c};
    \node[draw,circle] (alpha) at (8,3){$\alpha$};
    \node[draw,circle] (beta) at (8,2){$\beta$};
    \node[draw,circle] (gamma) at (8,1){$\gamma$};
    \node[draw=gray,fit=(1) (2) (3) (4) (5)] {};
    \node[draw=gray,fit=(a) (b) (c)] {};
    \node[draw=gray,fit=(a2) (b2) (c2)] {};
    \node[draw=gray,fit=(alpha) (beta) (gamma)] {};
    \seta{(1)}{(a)};
    \seta{(2)}{(c)};
    \seta{(3)}{(b)};
    \seta{(4)}{(b)};
    \seta{(5)}{(c)};
    \seta{(a2)}{(gamma)};
    \seta{(b2)}{(beta)};
    \seta{(c2)}{(alpha)};
  \end{tikzpicture}
  \end{center}
}

\myframe {
  \begin{center}
  \begin{tikzpicture}
    \node (f) at (1,5) {$f$};
    \node[draw,circle] (1) at (0,4){1};
    \node[draw,circle] (2) at (0,3){2};
    \node[draw,circle] (3) at (0,2){3};
    \node[draw,circle] (4) at (0,1){4};
    \node[draw,circle] (5) at (0,0){5};
    \node[draw,circle] (a) at (2,3){a};
    \node[draw,circle] (b) at (2,2){b};
    \node[draw,circle] (c) at (2,1){c};
    \node (g) at (3,5) {$g$};
    \node[draw,circle] (alpha) at (4,3){$\alpha$};
    \node[draw,circle] (beta) at (4,2){$\beta$};
    \node[draw,circle] (gamma) at (4,1){$\gamma$};
    \node[draw=gray,fit=(1) (2) (3) (4) (5)] {};
    \node[draw=gray,fit=(a) (b) (c)] {};
    \node[draw=gray,fit=(alpha) (beta) (gamma)] {};
    \seta{(1)}{(a)};
    \seta{(2)}{(c)};
    \seta{(3)}{(b)};
    \seta{(4)}{(b)};
    \seta{(5)}{(c)};
    \seta{(a)}{(gamma)};
    \seta{(b)}{(beta)};
    \seta{(c)}{(alpha)};
    \only<2>{\setaemph{(1)}{(a)}; \setaemph{(a)}{(gamma)};}
    \only<3>{\setaemph{(2)}{(c)}; \setaemph{(c)}{(alpha)};}
    \only<4>{\setaemph{(3)}{(b)}; \setaemph{(b)}{(beta)};}
    \only<5>{\setaemph{(4)}{(b)}; \setaemph{(b)}{(beta)};}
    \only<6>{\setaemph{(5)}{(c)}; \setaemph{(c)}{(alpha)};}
  \end{tikzpicture}
  \hspace{2 cm}
  \begin{tikzpicture}
    \node (f) at (1,5) {$g\circ f = g(f)$};
    \node[draw,circle] (1) at (0,4){1};
    \node[draw,circle] (2) at (0,3){2};
    \node[draw,circle] (3) at (0,2){3};
    \node[draw,circle] (4) at (0,1){4};
    \node[draw,circle] (5) at (0,0){5};
    \node[draw,circle] (alpha) at (2,3){$\alpha$};
    \node[draw,circle] (beta) at (2,2){$\beta$};
    \node[draw,circle] (gamma) at (2,1){$\gamma$};
    \only<2->{\seta{(1)}{(gamma)};}
    \only<3->{\seta{(2)}{(alpha)};}
    \only<4->{\seta{(3)}{(beta)};}
    \only<5->{\seta{(4)}{(beta)};}
    \only<6->{\seta{(5)}{(alpha)};}
  \end{tikzpicture}
  \end{center}
}

\myframe {
  $$ f:A\rightarrow B, \qquad g:B\rightarrow C $$
  $$ g\circ f:A\rightarrow C $$
  $$ (g\circ f)(x) = g(f(x)) $$
  Exemplo:
  \begin{itemize}
    \item $ f(x) = x^2 \qquad g(x) = x + 3 $
      $$ (g\circ f)(x) = g(f(x)) = f(x) + 3 = x^2 + 3 $$
      $$ (f\circ g)(x) = f(g(x)) = g(x)^2 = (x + 3)^2 $$
    \item $ f(x) = e^x \qquad g(x) = -x^2 $
      $$ (g\circ f)(x) = g(f(x)) = -f(x)^2 = -(e^x)^2 = -e^{2x} $$
      $$ (f\circ g)(x) = f(g(x)) = e^{g(x)} = e^{-x^2} $$
  \end{itemize}
}

\myframe {
  $$ f^2(x) = (f\circ f)(x) = f(f(x)) $$
  $$ f^3(x) = (f\circ f^2)(x) = f(f^2(x)) = f(f(f(x))) $$
  \spc
  \pause
  $$ {\color{red} f^2(x) \neq f(x)^2, \qquad \mbox{em geral}} $$
  Exemplo: $f(x) = x - 1$
    $$ f^2(x) = f(f(x)) = f(x) - 1 = x - 2 $$
    $$ f(x)^2 = (x - 1)^2 = x^2 - 2x + 1 $$
}

\makesubsection{Função inversa}

\myframe {
  Considere $f:A\rightarrow B$ bijetora.

  Então todo elemento de $B$ tem um único elemento associado em $A$.

  Então podemos definir essa função, que leva cada elemento de $B$ no
  seu elemento relacionado.

  Essa função é chamada de inversa, e é escrita como $f^{-1}:B \rightarrow A$.

  A inversa satisfaz $f(f^{-1}(y)) = y$ e $f^{-1}(f(x)) = x$.

  Exemplo:
  \begin{itemize}
    \item $f:\mathbb{Z}\rightarrow\mathbb{Z}, f(n) = n + 1$
      $$f^{-1}(n) = n - 1$$
    \item $f:[0,\infty)\rightarrow[0,\infty), f(x) = x^2$
      $$f^{-1}(x) = \sqrt{x}$$
  \end{itemize}
}

\myframe {
  Exemplo:
  \begin{itemize}
    \item $f:\mathbb{R}\rightarrow(0,\infty), 
      \qquad f(x) = e^x
      \qquad f^{-1}(x) = \ln x $
    \item $f:\mathbb{R}^*\rightarrow\mathbb{R}^*, 
      \qquad f(x) = \dfrac{1}{x}
      \qquad f^{-1}(x) = \dfrac{1}{x} $
    \item $f:(-\pi/2,\pi/2)\rightarrow\mathbb{R}, 
      \qquad f(x) = \tan x
      \qquad f^{-1}(x) = \mbox{arc}\tan x $
    \item $f:\mathbb{R}\rightarrow\mathbb{R}, 
      \qquad f(x) = mx + b, m \neq 0
      \qquad f^{-1}(x) = \dfrac{x - b}{m} $
    \item $f:\mathbb{N}\rightarrow\mathbb{Z},
      \qquad f(n) = \left\{
        \begin{array}{ll}
          -n/2, & n\mbox{ par} \\
          (n+1)/2 , & n\mbox{ ímpar}
        \end{array}\right.$
        $$f^{-1}(n) = \left\{
        \begin{array}{ll}
          -2n, & n \leq 0 \\
          2n - 1, & n > 0
        \end{array}\right.$$
  \end{itemize}
}


\end{document}

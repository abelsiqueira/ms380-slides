\documentclass[10 pt]{beamer}
\usepackage[brazil]{babel}
\usepackage[T1]{fontenc}
\usepackage{ae}
\usepackage[utf8]{inputenc}
\usepackage{framed}
%\usepackage[dvipsnames]{color}
\usepackage{graphicx}
\usepackage{algorithm}
\usepackage{algpseudocode}
\usepackage{epsfig}
\usepackage{abel}
\usepackage{tikz}
\usepackage{amssymb, amsmath, amsfonts}
\bibliographystyle{plain}
%\topmargin		0 cm
%\hoffset		0 cm
%\voffset		0 cm
%\evensidemargin		0 cm
%\oddsidemargin		0 cm
%\setlength{\textwidth}{16 cm}
%\setlength{\textheight}{21 cm}

\colorlet{mybasecolor}[rgb]{teal}

\colorlet{mydarkestcolor}[rgb]{mybasecolor!20!black}
\colorlet{mydarkcolor}[rgb]{mybasecolor!60!black}
\colorlet{mynormalcolor}[rgb]{mybasecolor!90!black}
\colorlet{mylightcolor}[rgb]{mybasecolor!10!white}

\colorlet{mysidebarcolor}[rgb]{mydarkcolor}
\colorlet{mylogobg}[rgb]{mylightcolor}
\colorlet{mylogofg}[rgb]{mydarkestcolor}
\colorlet{mysidebarauthor}[rgb]{mylightcolor}
\colorlet{mysidebartitle}[rgb]{mylightcolor}
\colorlet{mysidebarsecfg}[rgb]{mydarkestcolor}
\colorlet{mysidebarsecbg}[rgb]{mylightcolor}
\colorlet{mysidebarsubsecfg}[rgb]{mydarkestcolor}
\colorlet{mysidebarsubsecbg}[rgb]{mylightcolor}
\colorlet{myframetitlefg}[rgb]{mylightcolor}
\colorlet{myframetitlebg}[rgb]{mydarkcolor}
\colorlet{myitemcolor}[rgb]{mynormalcolor}
\colorlet{myfontcolor}[rgb]{mydarkestcolor}
\colorlet{mytitlefg}[rgb]{mylightcolor}
\colorlet{mytitlebg}[rgb]{mydarkcolor}
\colorlet{myauthorfg}[rgb]{mylightcolor}
\colorlet{myauthorbg}[rgb]{mydarkcolor}

\useinnertheme{rounded}
\setbeamercolor{structure}{fg = myitemcolor}
%\useoutertheme[height=1 cm,width=1.6 cm]{sidebar}
\useoutertheme{smoothtree}

\setbeamercolor{palette secondary}{bg=mylogobg,fg=mylogofg} %cor do logo
\setbeamercolor{sidebar}{bg=mysidebarcolor} %Background do sidebar
\setbeamercolor{palette sidebar tertiary}{fg = mysidebarauthor} %Autor no sidebar
\setbeamercolor{palette sidebar quaternary}{fg = mysidebartitle} %Titulo no sidebar
\setbeamercolor{section in sidebar}{fg = mysidebarsecfg,bg = mysidebarsecbg} %A cor eh a media de fg e bg
\setbeamercolor{subsection in sidebar}{fg = mysidebarsubsecfg,bg = mysidebarsubsecbg} %A cor eh a media de fg e bg
\setbeamercolor{frametitle}{fg = myframetitlefg, bg = myframetitlebg} 
\setbeamercolor{title}{bg = mytitlebg,fg = mytitlefg}
\setbeamerfont{title}{series = \bf}
\setbeamercolor{author}{bg = myauthorbg,fg = myauthorfg}
\setbeamerfont{author}{series = \it}
\setbeamercolor{normal text}{bg = white, fg = myfontcolor}
%\setbeamercolor{normal text}{bg = black, fg = white}
%\setbeamerfont{normal text}{family=serif}
\usefonttheme[onlymath]{serif}
%\logo{\includegraphics[scale=0.35]{logo.png}}
%\logo{$\min f(x)$}

\newcommand{\makesection}[1]{\section[#1]{#1}}
\newcommand{\makesubsection}[1]{\subsection[#1]{#1}}

\algrenewcommand\algorithmicif{\textbf{se}}
\algrenewcommand\algorithmicthen{\textbf{então}}
\algrenewcommand\algorithmicend{\textbf{fim do}}
\algrenewcommand\algorithmicwhile{\textbf{enquanto}}
\algrenewcommand\algorithmicdo{\textbf{faça}}
\algrenewcommand\algorithmicelse{\textbf{senão}}

\title{ MS380 - Aula Complementar \\
Equações Lineares, Quadráticas e Polinômios}
\author{Abel Soares Siqueira \\
Laércio Luís Vendite}
\date{}

\newcommand{\myframe}[1]{
\begin{frame}
 \frametitle{\insertsection \qquad {\small \insertsubsection}}
#1
\end{frame}}
\newcommand{\myframetop}[1]{
\begin{frame}[t]
 \frametitle{\insertsection \qquad {\small \insertsubsection}}
#1
\end{frame}}
% \newcommand{\visiblelbl}[1]{\label{#1}{\color{red}{#1}}}
\newcommand{\visiblelbl}[1]{\label{#1}}
\newcommand{\spc}{\vspace{0.5 cm}}
\newcommand{\systemtwo}[1]{$
  \left\{\begin{array}{rcrcr} #1
  \end{array}\right.$}
\newcommand{\systemthree}[1]{$
  \left\{\begin{array}{rcrcrcr}
  #1
  \end{array}\right.$}

\begin{document}

\begin{frame}
 \titlepage
\end{frame}

\makesection{Equações Lineares}

\myframe{
  $$y = mx + b, \qquad m \neq 0$$
  \begin{center}
  \begin{tikzpicture}[domain=-0.1:5,scale=0.7]
    \draw[very thin, color=gray] (-0.1, -0.1) grid (4.9, 4.9);
    \draw[->] (-0.2,0) -- (5.2,0) node[right] {$x$};
    \draw[->] (0,-0.2) -- (0,5.2) node[above] {$y$};
    \draw[color=red] plot function{x/2+1}
      node[above] {$y = mx + b$};
    \node[color=red,left] (b) at (0,1) {$b$};
    \pause
    \draw[color=blue,dashed] 
      (2,0) node[below] {$x_0$} -- (2,2) -- (0,2) node[left] {$y_0$};
    \draw[color=blue,dashed] 
      (4,0) node[below] {$x_1$} -- (4,3) -- (0,3) node[left] {$y_1$};
    \pause
    \draw[color=blue]
      (2,2) -- (4,2) node[midway,below] {$x_1-x_0$} -- (4,3)node[midway,right] {$y_1-y_0$};
    \draw[color=blue] (2,2) + (0.6,0) arc (0:26:0.6);
  \end{tikzpicture}
  \end{center}
  \onslide<4->{
  $$m = \frac{y_1 - y_0}{x_1 - x_0}, \qquad \qquad
  y - y_0 = m(x - x_0).$$}
}

\makesection{Equações Quadráticas}

\myframe {
  $$ax^2 + bx + c = 0, \qquad a \neq 0$$
  \begin{center}
  \begin{tikzpicture}[domain=-0.5:3]
    \draw[very thin, color=gray] (-0.8, -1.4) grid(3.5, 3.5);
    \draw[->] (-0.9,0) -- (3.7,0) node[right] {$x$};
    \draw[->] (0,-1.5) -- (0,3.7) node[above] {$y$};
    \draw[color=red] plot (\x, \x*\x-2*\x-1/5)
      node[right] {$y = ax^2 + bx + c$};
    \draw[color=red] (0.1, -1/5) -- (-0.1, -1/5)
      node[left] {$c$};
  \end{tikzpicture}
  \end{center}
}

\makesubsection{Discriminante}

\myframe {
  $$ax^2 + bx + c = 0, \qquad a \neq 0$$
  \pause
  $$ \Delta = b^2 - 4ac $$
  \pause
  \begin{description}
    \item<3->[$\Delta > 0$] Duas soluções distintas.
      $$ x = \frac{-b \pm \sqrt{\Delta}}{2a}. $$
    \item<4->[$\Delta = 0$] Uma solução.
      $$ x = \frac{-b}{2a}. $$
    \item<5->[$\Delta < 0$] Não existe solução real.
  \end{description}
}

\myframe {
  $$ax^2 + bx + c = 0, \qquad a \neq 0$$
  \begin{center}
    Vértice: $(x_v, y_v)$
  \end{center}
  $$ x_v = \frac{-b}{2a} \qquad y_v = \frac{-\Delta}{4a} $$
  \pause
  \begin{center}
    Concavidade: \spc \\
    \begin{tikzpicture}[domain=-2:2,scale=0.5]
      \draw[color=red] plot (\x, \x*\x);
      \node[color=red] at (0,2) {$a > 0$};
    \end{tikzpicture}\hspace{1 cm}
    \begin{tikzpicture}[domain=-2:2,scale=0.5]
      \draw[color=red] plot (\x, -\x*\x);
      \node[color=red] at (0,-2) {$a < 0$};
    \end{tikzpicture}
  \end{center}
}

\makesubsection{Resumo visual}

\myframe {
  $$ \Delta > 0 $$
  \spc
  \begin{center}
    \begin{tikzpicture}[domain=-2.09:2.09,scale=0.5]
      \draw[->] (-2,1) -- (2,1) node[right] {$x$};
      \draw[color=red] plot (\x, \x*\x);
      \node[color=red] at (0,2) {$a > 0$};
    \end{tikzpicture}\hspace{1 cm}
    \begin{tikzpicture}[domain=-2.09:2.09,scale=0.5]
      \draw[->] (-2,-1) -- (2,-1) node[right] {$x$};
      \draw[color=red] plot (\x, -\x*\x);
      \node[color=red] at (0,-2) {$a < 0$};
    \end{tikzpicture}
  \end{center}
}

\myframe {
  $$ \Delta = 0 $$
  \spc
  \begin{center}
    \begin{tikzpicture}[domain=-2:2,scale=0.5]
      \draw[->] (-2,0) -- (2,0) node[right] {$x$};
      \draw[color=red] plot (\x, \x*\x);
      \node[color=red] at (0,2) {$a > 0$};
    \end{tikzpicture}\hspace{1 cm}
    \begin{tikzpicture}[domain=-2:2,scale=0.5]
      \draw[->] (-2,0) -- (2,0) node[right] {$x$};
      \draw[color=red] plot (\x, -\x*\x);
      \node[color=red] at (0,-2) {$a < 0$};
    \end{tikzpicture}
  \end{center}
}

\myframe {
  $$ \Delta < 0 $$
  \spc
  \begin{center}
    \begin{tikzpicture}[domain=-1.73:1.73,scale=0.5]
      \draw[->] (-2,-1) -- (2,-1) node[right] {$x$};
      \draw[color=red] plot (\x, \x*\x);
      \node[color=red] at (0,2) {$a > 0$};
    \end{tikzpicture}\hspace{1 cm}
    \begin{tikzpicture}[domain=-1.73:1.73,scale=0.5]
      \draw[->] (-2,1) -- (2,1) node[right] {$x$};
      \draw[color=red] plot (\x, -\x*\x);
      \node[color=red] at (0,-2) {$a < 0$};
    \end{tikzpicture}
  \end{center}
}

\myframe {
  $$ y = ax^2 + bx + c $$
  \pause
  \begin{center}
  Se $\Delta \geq 0$, as raízes são $x_0$ e $x_1$.
  \end{center}
  \pause
  \begin{align*}
    \onslide<3->{ y & = a(x - x_0)(x - x_1) } \\
    \onslide<4->{ y & = ax^2 - ax(x_0+x_1) + ax_0x_1 }
  \end{align*}
  \onslide<5->{
  $$ x_0 + x_1 = -\frac{b}{a} \qquad x_0x_1 = \frac{c}{a}. $$}
}

\makesection{Polinômios}

\myframe {
    $$ p_n(x) = a_0 + a_1x + a_2x^2 + \dots + a_nx^n. $$
  \begin{itemize}
    \item Se o grau é ímpar, existe ao menos uma raiz real.
    \item Se uma raiz é complexa, o conjugado dessa raiz é outra raiz.
    \item Se $x_1, x_2, \dots, x_n$ são as raízes (podendo ser complexas),
      $$ p_n(x) = a_n(x - x_1)(x - x_2)\dots(x - x_n). $$
  \end{itemize}
}

\makesubsection{Equações de Girard}

\myframe {
  $$ax^3 + bx^2 + cx + d = 0$$
  \spc\pause
  $$ x_1 + x_2 + x_3 = \frac{-b}{a} \qquad
     x_1x_2 + x_1x_3 + x_2x_3 = \frac{c}{a} \qquad
     x_1x_2x_3 = -\frac{d}{a} $$
}

\end{document}

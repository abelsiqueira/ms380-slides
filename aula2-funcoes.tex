\documentclass[10 pt]{beamer}
\usepackage[brazil]{babel}
\usepackage[T1]{fontenc}
\usepackage{ae}
\usepackage[utf8]{inputenc}
\usepackage{framed}
%\usepackage[dvipsnames]{color}
\usepackage{graphicx}
\usepackage{algorithm}
\usepackage{algpseudocode}
\usepackage{epsfig}
\usepackage{tikz}
\usepackage{pgfplots}
\usepackage{amssymb, amsmath, amsfonts}
\bibliographystyle{plain}
%\topmargin		0 cm
%\hoffset		0 cm
%\voffset		0 cm
%\evensidemargin		0 cm
%\oddsidemargin		0 cm
%\setlength{\textwidth}{16 cm}
%\setlength{\textheight}{21 cm}

\colorlet{mybasecolor}[rgb]{teal}

\colorlet{mydarkestcolor}[rgb]{mybasecolor!20!black}
\colorlet{mydarkcolor}[rgb]{mybasecolor!60!black}
\colorlet{mynormalcolor}[rgb]{mybasecolor!90!black}
\colorlet{mylightcolor}[rgb]{mybasecolor!10!white}

\colorlet{mysidebarcolor}[rgb]{mydarkcolor}
\colorlet{mylogobg}[rgb]{mylightcolor}
\colorlet{mylogofg}[rgb]{mydarkestcolor}
\colorlet{mysidebarauthor}[rgb]{mylightcolor}
\colorlet{mysidebartitle}[rgb]{mylightcolor}
\colorlet{mysidebarsecfg}[rgb]{mydarkestcolor}
\colorlet{mysidebarsecbg}[rgb]{mylightcolor}
\colorlet{mysidebarsubsecfg}[rgb]{mydarkestcolor}
\colorlet{mysidebarsubsecbg}[rgb]{mylightcolor}
\colorlet{myframetitlefg}[rgb]{mylightcolor}
\colorlet{myframetitlebg}[rgb]{mydarkcolor}
\colorlet{myitemcolor}[rgb]{mynormalcolor}
\colorlet{myfontcolor}[rgb]{mydarkestcolor}
\colorlet{mytitlefg}[rgb]{mylightcolor}
\colorlet{mytitlebg}[rgb]{mydarkcolor}
\colorlet{myauthorfg}[rgb]{mylightcolor}
\colorlet{myauthorbg}[rgb]{mydarkcolor}

\useinnertheme{rounded}
\setbeamercolor{structure}{fg = myitemcolor}
%\useoutertheme[height=1 cm,width=1.6 cm]{sidebar}
\useoutertheme{smoothtree}

\setbeamercolor{palette secondary}{bg=mylogobg,fg=mylogofg} %cor do logo
\setbeamercolor{sidebar}{bg=mysidebarcolor} %Background do sidebar
\setbeamercolor{palette sidebar tertiary}{fg = mysidebarauthor} %Autor no sidebar
\setbeamercolor{palette sidebar quaternary}{fg = mysidebartitle} %Titulo no sidebar
\setbeamercolor{section in sidebar}{fg = mysidebarsecfg,bg = mysidebarsecbg} %A cor eh a media de fg e bg
\setbeamercolor{subsection in sidebar}{fg = mysidebarsubsecfg,bg = mysidebarsubsecbg} %A cor eh a media de fg e bg
\setbeamercolor{frametitle}{fg = myframetitlefg, bg = myframetitlebg} 
\setbeamercolor{title}{bg = mytitlebg,fg = mytitlefg}
\setbeamerfont{title}{series = \bf}
\setbeamercolor{author}{bg = myauthorbg,fg = myauthorfg}
\setbeamerfont{author}{series = \it}
\setbeamercolor{normal text}{bg = white, fg = myfontcolor}
%\setbeamercolor{normal text}{bg = black, fg = white}
%\setbeamerfont{normal text}{family=serif}
\usefonttheme[onlymath]{serif}
%\logo{\includegraphics[scale=0.35]{logo.png}}
%\logo{$\min f(x)$}

\newcommand{\makesection}[1]{\section[#1]{#1}}
\newcommand{\makesubsection}[1]{\subsection[#1]{#1}}

\algrenewcommand\algorithmicif{\textbf{se}}
\algrenewcommand\algorithmicthen{\textbf{então}}
\algrenewcommand\algorithmicend{\textbf{fim do}}
\algrenewcommand\algorithmicwhile{\textbf{enquanto}}
\algrenewcommand\algorithmicdo{\textbf{faça}}
\algrenewcommand\algorithmicelse{\textbf{senão}}

\title{ MS380 - Aula 2 \\
Funções}
\author{Abel Soares Siqueira \\
Laércio Luís Vendite}
\date{}

\newcommand{\myframe}[1]{
\begin{frame}
 \frametitle{\insertsection \qquad {\small \insertsubsection}}
#1
\end{frame}}
\newcommand{\myframetop}[1]{
\begin{frame}[t]
 \frametitle{\insertsection \qquad {\small \insertsubsection}}
#1
\end{frame}}
% \newcommand{\visiblelbl}[1]{\label{#1}{\color{red}{#1}}}
\newcommand{\visiblelbl}[1]{\label{#1}}
\newcommand{\spc}{\vspace{0.5 cm}}
\newcommand{\systemtwo}[1]{$
  \left\{\begin{array}{rcrcr} #1
  \end{array}\right.$}
\newcommand{\systemthree}[1]{$
  \left\{\begin{array}{rcrcrcr}
  #1
  \end{array}\right.$}

\begin{document}

\begin{frame}
 \titlepage
\end{frame}

\makesection{Funções}

\myframe {
  Uma função associa elementos de dois conjuntos, sendo que cada
  elemento do primeiro conjunto tem um único elemento do segundo
  conjunto associado a ele.
  \pause
  $$f:A\rightarrow B$$
  \pause
  $A$ é o domínio (o primeiro conjunto). \\
  \pause
  $B$ é o contra-domínio (o segundo conjunto).

  Exemplo:

  $A = \{1,a,\mbox{laranja}\}$ e $B = \{2,3,b,\mbox{verde},\mbox{MS380}\}$.

  \begin{center}
    \begin{tabular}{c|c}
      {\bf Domínio} & {\bf Contra-Domínio} \\ \hline
      1 & b \\
      a & 3 \\
      \mbox{laranja} & \mbox{MS380}
    \end{tabular}
  \end{center}

  $f(1) = b$, $f(a) = 3$, $f(\mbox{laranja}) = \mbox{MS380}$.
}

\myframe {
  Exemplo:

  $$f:\mathbb{N}\rightarrow\mathbb{N}, \qquad f(n) = n + 1.$$

  $$f:\mathbb{R}\rightarrow\mathbb{R}, \qquad f(x) = x^2.$$

}

\myframe {
  {\color{red} Não são funções:}

  $$f:\mathbb{N}\rightarrow\mathbb{N}, \qquad f(n) = n - 1.$$
  Não é função porque $f(1)$ não está bem definido ($0 \not\in \mathbb{N}$).

  $$f:\mathbb{R}\rightarrow\R, \qquad f(x) = \sqrt{x}.$$
  Não é função porque raiz de número negativo não está definida nos reais.
}

\makesubsection{Imagem}

\myframe {
  A imagem da função $f:A\rightarrow B$ é um subconjunto do contra-domínio com todas os
  valores de $f$ nos elementos do domínio.
  $$ Im(f) = \{ f(x) : x \in A \}. $$

Exemplo:
  $$ f:\{1,2,3\}\rightarrow\{a,b,c,d\} $$
  $$ f(1) = a, f(2) = d, f(3) = b. $$
  $$ Im(f) = \{f(1),f(2),f(3)\} = \{a,b,d\}. $$

Exemplo:
  $$ f:\mathbb{N}\rightarrow\mathbb{N}, \qquad f(n) = n + 1 $$
  $$ Im(f) = \{f(1),f(2),f(3),\dots\} = \{2,3,4,\dots\} = \mathbb{N}-\{1\}.$$
}

\makesubsection{Domínio Máximo}

\myframe {
  Quando definimos uma função $f:A\rightarrow B$, seu domínio $A$ às vezes
  pode ser estendido. Procuramos pelo domínio máximo que $f$ pode ter,
  e ajustamos o contra-domínio adequadamente.

  Também podemos pedir o domínio máximo com o contra-domínio fixado.

\vspace{0.5 cm}
Exemplo:
  $$ f:\mathbb{N} \rightarrow\mathbb{N}, \qquad f(n) = n + 1. $$
  Considerando apenas os reais (sem os complexos), podemos estender
  essa função para todos os reais, e o contra-domínio também precisa
  ser estendido para todos os reais.

  Se mantemos o contra-domínio fixo, podemos estender o domínio
  para $\mathbb{N}\cup\{0\}$.
}

\myframe {
  Às vezes damos apenas a definição da função.

\spc\pause
Exemplo:
  $$ f(x) = \sqrt{x} $$
  A raiz quadrada só é definida para todos os número não-negativos. Então
  o domínio é $D(f) = [0,\infty) = \mathbb{R}_+$.

\spc\pause
Exemplo:
  $$ f(x) = \frac{\sqrt{1 - x^2}}{x}. $$
  Devemos ter $1 - x^2 \geq 0$ porque a raiz só é definida
  para números não-negativos, e devemos ter $x \neq 0$,
  porque a divisão por zero não existe.
  $$ 1 - x^2 \geq 0 \Rightarrow x^2 \leq 1 \Rightarrow -1 \leq x \leq 1 $$
  Então devemos $D(f) = \{x : -1 \leq x \leq 1\mbox{ e }x \neq 0\} =
  [-1,0)\cup(0,1]$.
}

\makesubsection{Gráficos}

\myframe {
  O gráfico de uma função é a representação da mesma num plano cartesiano.
  Para cada valor $x$, determinamos um valor $y = f(x)$, e obtemos o
  ponto $(x,y)$.

\spc\pause
Exemplo: $$ f:[-1,2]\rightarrow\R, \qquad f(x) = -x^2 + x. $$
  \begin{center}
  \begin{tikzpicture}[domain=-1:2]
    \draw[very thin, color=gray] (-1.2, -2.2) grid (2.2, 1.2);
    \draw[->] (-1.2,0) -- (2.2,0) node[right] {$x$};
    \draw[->] (0,-2.2) -- (0,1.2) node[above] {$f(x)$};
    \draw[color=red] plot (\x, -\x*\x + \x);
  \end{tikzpicture}
  \end{center}
}

\myframe {
Exemplo: $$ f(x) = \sqrt{1 - x^4}. $$
  \begin{center}
  \begin{tikzpicture}[domain=-1:1]
    \draw[very thin, color=gray] (-1.2, -0.2) grid (1.2, 1.2);
    \draw[->] (-1.2,0) -- (1.2,0) node[right] {$x$};
    \draw[->] (0,-0.2) -- (0,1.2) node[above] {$f(x)$};
    \draw[color=red] plot (\x, 1 - \x*\x*\x*\x);
  \end{tikzpicture}
  \end{center}
}

\myframe {
Exemplo: $$ f(x) = \frac{1}{2}\vert x^2 - 4x + 3\vert. $$
  \begin{center}
  \begin{tikzpicture}[samples at={-0.1,0,...,4.2}]
    \draw[very thin, color=gray] (-0.2, -0.2) grid (4.2, 2.2);
    \draw[->] (0.0,0) -- (4.2,0) node[right] {$x$};
    \draw[->] (0,-0.2) -- (0,2.2) node[above] {$f(x)$};
    \draw[color=red] plot (\x, {0.5*abs(\x*\x - 4*\x + 3)});
  \end{tikzpicture}
  \end{center}
}

\makesubsection{Funções Elementares}

\myframe {
  {\bf Função constante: $f(x) = c$}
  \begin{center}
  \begin{tikzpicture}[domain=-3.2:3.2]
    \draw[very thin, color=gray] (-3.2, -0.2) grid (3.2, 1.2);
    \draw[->] (-3.2,0) -- (3.2,0) node[right] {$x$};
    \draw[->] (0,-0.2) -- (0,1.2) node[above] {$f(x)$};
    \draw[color=red] plot (\x, 0.7);
    \node[color=red,below left] at (0,0.7) {$c$};
  \end{tikzpicture}
  \end{center}
}

\myframe {
  {\bf Função linear: $f(x) = mx + b$, $m \neq 0$}
  \begin{center}
  \begin{tikzpicture}[domain=-2.8:1.2]
    \draw[very thin, color=gray] (-2.8, -1.2) grid (1.2, 2.2);
    \draw[->] (-2.8,0) -- (1.2,0) node[right] {$x$};
    \draw[->] (0,-1.2) -- (0,2.2) node[above] {$f(x)$};
    \draw[color=red] plot (\x, 0.6*\x + 1.2);
    \node[color=red,above left] at (0,1.2) {$b$};
  \end{tikzpicture}
  \end{center}
}

\myframe {
  {\bf Função quadrática: $f(x) = ax^2 + bx + c$, $a \neq 0$}
  \begin{center}
  \begin{tikzpicture}[domain=-2.8:1.2]
    \draw[very thin, color=gray] (-2.8, -2.2) grid (1.2, 2.2);
    \draw[->] (-2.8,0) -- (1.2,0) node[right] {$x$};
    \draw[->] (0,-2.2) -- (0,2.2) node[above] {$f(x)$};
    \draw[color=red] plot (\x, 0.6*\x*\x + 0.4*\x - 1.8);
    \node[color=red,above left] at (0,-1.8) {$c$};
  \end{tikzpicture}
  \end{center}
}

\myframe {
  {\bf Função Potência: $f(x) = x^{\alpha}$, $a \neq 0$}
  \spc \\
  \begin{center}
  \begin{tikzpicture}[domain=-1.2:1.2]
    \draw[very thin, color=gray] (-1.2, -0.1) grid (1.2, 2.2);
    \draw[->] (-1.2,0) -- (1.2,0) node[right] {$x$};
    \draw[->] (0,-0.1) -- (0,2.2) node[above] {$f(x)$};
    \draw[color=red] plot (\x, \x*\x*\x*\x);
    \node[color=red,below] at (0,0) {$a = 5$};
  \end{tikzpicture}
  \hspace{0.5 cm}
  \begin{tikzpicture}[domain=0.0:2.2]
    \draw[very thin, color=gray] (-0.2, -0.1) grid (2.2, 2.2);
    \draw[->] (-0.2,0) -- (2.2,0) node[right] {$x$};
    \draw[->] (0,-0.1) -- (0,2.2) node[above] {$f(x)$};
    \draw[color=red] plot (\x, {sqrt(\x)});
    \node[color=red,below] at (1,0) {$a = 0.5$};
  \end{tikzpicture}
  \hspace{0.5 cm}
  \begin{tikzpicture}
    \draw[very thin, color=gray] (-1.2, -2.1) grid (1.2, 2.2);
    \draw[->] (-1.2,0) -- (1.2,0) node[right] {$x$};
    \draw[->] (0,-2.1) -- (0,2.2) node[above] {$f(x)$};
    \draw[color=red,domain=-1.2:-0.4] plot (\x, {1/\x});
    \draw[color=red,domain=0.4:1.2] plot (\x, {1/\x});
    \node[color=red,below right] at (0,0) {$a = -1$};
  \end{tikzpicture}
  \end{center}
}

\myframe {
  {\bf Função exponencial: $f(x) = a^x$, $a > 0, a\neq 1$}
  \begin{center}
  \begin{tikzpicture}[domain=-2.2:2.2]
    \draw[very thin, color=gray] (-2.2, -0.2) grid (2.2, 3.2);
    \draw[->] (-2.2,0) -- (2.2,0) node[right] {$x$};
    \draw[->] (0,-0.2) -- (0,3.2) node[above] {$f(x)$};
    \draw[color=red] plot (\x, {1.6^\x});
    \draw[color=red] (0.1,1) -- (-0.1,1)
      node[left] {$1$};
    \draw[color=blue,dashed] (1,0) node[below] {$1$} 
      -- (1,1.6) -- (0,1.6) node[left] {$a$};
    \draw[color=blue,dashed] (-1,0) node[below] {$-1$} 
      -- (-1,0.625) -- (0,0.625) node[right] {$1/a$};
  \end{tikzpicture}
  \end{center}
  Tradicionalmente com $a = e \approx 2.7183$.
}

\myframe {
  {\bf Função logarítmica: $f(x) = \log_a x$, $a > 0, a\neq 1$}
  \begin{center}
  \begin{tikzpicture}[domain=0.1:4.2]
    \draw[very thin, color=gray] (-0.2, -2.2) grid (4.2, 2.2);
    \draw[->] (-0.2,0) -- (4.2,0) node[right] {$x$};
    \draw[->] (0,-2.2) -- (0,2.2) node[above] {$f(x)$};
    \draw[color=red] plot (\x, {ln(\x)});
    \draw[color=red] (1,0.1) -- (1,-0.1)
      node[below] {$1$};
    \draw[color=blue,dashed] (2.718,0) node[below] {$a$} 
      -- (2.718,1) -- (0,1) node[left] {$1$};
    \draw[color=blue,dashed] (0.368,0) node[above] {$1/a$} 
      -- (0.368,-1) -- (0,-1) node[left] {$-1$};
  \end{tikzpicture}
  \end{center}
  $$ \ln(x) = \log_e(x) $$
}

\myframe {
  {\bf Função seno: $f(x) = \sin(x)$ }
  \begin{center}
  \begin{tikzpicture}[domain=-1.2:7.2]
    \draw[very thin, color=gray] (-1.2, -1.2) grid (7.2, 1.2);
    \draw[->] (-1.2,0) -- (7.2,0) node[right] {$x$};
    \draw[->] (0,-1.2) -- (0,1.2) node[above] {$f(x)$};
    \draw[color=red] plot (\x, {sin(\x r)});
    \draw[color=blue,dashed] (-1.2,1) -- (0,1) 
      node[above left] {$1$} -- (7.2,1);
    \draw[color=blue,dashed] (-1.2,-1) -- (0,-1) 
      node[below left] {$-1$} -- (7.2,-1);
    \draw[color=blue,dashed] (1.571,0) node[below] {$\frac{\pi}{2}$} 
        -- (1.571,1);
    \draw[color=blue,dashed] (3.1416 + 1.571,0) node[above] {$\frac{3\pi}{2}$} 
        -- (3.1416 + 1.571,-1);
    \draw[color=blue] (0,0.1) -- (0,-0.1) node[midway,below right] {$0$};
    \draw[color=blue] (3.1416,-0.1) -- (3.1416,0.1) node[above] {$\pi$};
    \draw[color=blue] (6.2832,0.1) -- (6.2832,-0.1) node[below] {$2\pi$};
  \end{tikzpicture}
  \end{center}
}

\myframe {
  {\bf Função cosseno: $f(x) = \cos(x)$ }
  \begin{center}
  \begin{tikzpicture}[domain=-1.2:7.2]
    \draw[very thin, color=gray] (-1.2, -1.2) grid (7.2, 1.2);
    \draw[->] (-1.2,0) -- (7.2,0) node[right] {$x$};
    \draw[->] (0,-1.2) -- (0,1.2) node[above] {$f(x)$};
    \draw[color=red] plot (\x, {cos(\x r)});
    \draw[color=blue,dashed] (-1.2,1) -- (0,1) 
      node[above left] {$1$} -- (7.2,1);
    \draw[color=blue,dashed] (-1.2,-1) -- (0,-1) 
      node[below left] {$-1$} -- (7.2,-1);
    \draw[color=blue,dashed] (6.2832,0) node[below] {$2\pi$} 
        -- (6.2832,1);
    \draw[color=blue,dashed] (3.1416,0) node[above] {$\pi$} 
        -- (3.1416,-1);
    \node[color=blue] at (0,0) [below right] {$0$};
    \draw[color=blue] (3.1416/2,-0.1) -- (3.1416/2,0.1) node[above] {$\frac{\pi}{2}$};
    \draw[color=blue] (3.1416*3/2,0.1) -- (3.1416*3/2,-0.1) node[below] 
      {$\frac{3\pi}{2}$};
  \end{tikzpicture}
  \end{center}
}

\myframe {
  {\bf Função tangente: $f(x) = \tan(x) = \frac{\sin(x)}{\cos(x)}$ }
  \begin{center}
  \begin{tikzpicture}[domain=-1.26:1.26]
    \draw[very thin, color=gray] (-1.8,-3.1) grid(1.8,3.1);
    \draw[->] (-1.8,0) -- (1.8,0) node[right] {$x$};
    \draw[->] (0,-3.1) -- (0,3.1) node[above] {$f(x)$};
    \draw[color=red] plot (\x, {tan(\x r)});
    \draw[color=blue,dashed] (-1.56,-3.1) -- (-1.56,3.1)
      node[midway,below left] {$-\frac{\pi}{2}$};
    \draw[color=blue,dashed] (1.56,-3.1) -- (1.56,3.1)
      node[midway,below right] {$\frac{\pi}{2}$};
  \end{tikzpicture}
  \end{center}
}

\end{document}

\documentclass[10 pt]{beamer}
\usepackage[brazil]{babel}
\usepackage[T1]{fontenc}
\usepackage{ae}
\usepackage[utf8]{inputenc}
\usepackage{framed}
%\usepackage[dvipsnames]{color}
\usepackage{graphicx}
\usepackage{algorithm}
\usepackage{algpseudocode}
\usepackage{epsfig}
\usepackage{abel}
\usepackage{tikz}
\usepackage{amssymb, amsmath, amsfonts}
\bibliographystyle{plain}
%\topmargin		0 cm
%\hoffset		0 cm
%\voffset		0 cm
%\evensidemargin		0 cm
%\oddsidemargin		0 cm
%\setlength{\textwidth}{16 cm}
%\setlength{\textheight}{21 cm}

\colorlet{mybasecolor}[rgb]{teal}

\colorlet{mydarkestcolor}[rgb]{mybasecolor!20!black}
\colorlet{mydarkcolor}[rgb]{mybasecolor!60!black}
\colorlet{mynormalcolor}[rgb]{mybasecolor!90!black}
\colorlet{mylightcolor}[rgb]{mybasecolor!10!white}

\colorlet{mysidebarcolor}[rgb]{mydarkcolor}
\colorlet{mylogobg}[rgb]{mylightcolor}
\colorlet{mylogofg}[rgb]{mydarkestcolor}
\colorlet{mysidebarauthor}[rgb]{mylightcolor}
\colorlet{mysidebartitle}[rgb]{mylightcolor}
\colorlet{mysidebarsecfg}[rgb]{mydarkestcolor}
\colorlet{mysidebarsecbg}[rgb]{mylightcolor}
\colorlet{mysidebarsubsecfg}[rgb]{mydarkestcolor}
\colorlet{mysidebarsubsecbg}[rgb]{mylightcolor}
\colorlet{myframetitlefg}[rgb]{mylightcolor}
\colorlet{myframetitlebg}[rgb]{mydarkcolor}
\colorlet{myitemcolor}[rgb]{mynormalcolor}
\colorlet{myfontcolor}[rgb]{mydarkestcolor}
\colorlet{mytitlefg}[rgb]{mylightcolor}
\colorlet{mytitlebg}[rgb]{mydarkcolor}
\colorlet{myauthorfg}[rgb]{mylightcolor}
\colorlet{myauthorbg}[rgb]{mydarkcolor}

\useinnertheme{rounded}
\setbeamercolor{structure}{fg = myitemcolor}
%\useoutertheme[height=1 cm,width=1.6 cm]{sidebar}
\useoutertheme{smoothtree}

\setbeamercolor{palette secondary}{bg=mylogobg,fg=mylogofg} %cor do logo
\setbeamercolor{sidebar}{bg=mysidebarcolor} %Background do sidebar
\setbeamercolor{palette sidebar tertiary}{fg = mysidebarauthor} %Autor no sidebar
\setbeamercolor{palette sidebar quaternary}{fg = mysidebartitle} %Titulo no sidebar
\setbeamercolor{section in sidebar}{fg = mysidebarsecfg,bg = mysidebarsecbg} %A cor eh a media de fg e bg
\setbeamercolor{subsection in sidebar}{fg = mysidebarsubsecfg,bg = mysidebarsubsecbg} %A cor eh a media de fg e bg
\setbeamercolor{frametitle}{fg = myframetitlefg, bg = myframetitlebg} 
\setbeamercolor{title}{bg = mytitlebg,fg = mytitlefg}
\setbeamerfont{title}{series = \bf}
\setbeamercolor{author}{bg = myauthorbg,fg = myauthorfg}
\setbeamerfont{author}{series = \it}
\setbeamercolor{normal text}{bg = white, fg = myfontcolor}
%\setbeamercolor{normal text}{bg = black, fg = white}
%\setbeamerfont{normal text}{family=serif}
\usefonttheme[onlymath]{serif}
%\logo{\includegraphics[scale=0.35]{logo.png}}
%\logo{$\min f(x)$}

\newcommand{\makesection}[1]{\section[#1]{#1}}
\newcommand{\makesubsection}[1]{\subsection[#1]{#1}}

\algrenewcommand\algorithmicif{\textbf{se}}
\algrenewcommand\algorithmicthen{\textbf{então}}
\algrenewcommand\algorithmicend{\textbf{fim do}}
\algrenewcommand\algorithmicwhile{\textbf{enquanto}}
\algrenewcommand\algorithmicdo{\textbf{faça}}
\algrenewcommand\algorithmicelse{\textbf{senão}}

\title{ MS380 - Aula Complementar \\
Sistemas Lineares}
\author{Abel Soares Siqueira \\
Laércio Luís Vendite}
\date{}

\newcommand{\myframe}[1]{
\begin{frame}
 \frametitle{\insertsection \qquad {\small \insertsubsection}}
#1
\end{frame}}
\newcommand{\myframetop}[1]{
\begin{frame}[t]
 \frametitle{\insertsection \qquad {\small \insertsubsection}}
#1
\end{frame}}
% \newcommand{\visiblelbl}[1]{\label{#1}{\color{red}{#1}}}
\newcommand{\visiblelbl}[1]{\label{#1}}
\newcommand{\spc}{\vspace{0.5 cm}}
\newcommand{\systemtwo}[1]{$
  \left\{\begin{array}{rcrcr} #1
  \end{array}\right.$}
\newcommand{\systemthree}[1]{$
  \left\{\begin{array}{rcrcrcr}
  #1
  \end{array}\right.$}

\begin{document}

\begin{frame}
 \titlepage
\end{frame}

\makesection{Sistemas Lineares}
\myframe {
\begin{center}
  \systemtwo{
    a_{11}x & + & a_{12}y & = & b_1 \\
    a_{21}x & + & a_{22}y & = & b_2
  }
  \spc
  \begin{itemize}
    \item Substituição
    \item Eliminação
  \end{itemize}
  \end{center}
}

\makesubsection{Substituição}
\myframe {
\begin{center}
  \systemtwo{
    3x & + & 5y & = & 8 \\
    4x & - & 3y & = & 1
  } \\
    \pause
    $$ 3x = 8 - 5y \Longrightarrow
    x  = \frac{8}{3} - \frac{5}{3}y $$
    \pause
    $$ 4\bigg(\frac{8}{3} - \frac{5}{3}y\bigg) - 3y = 1 
      \Longrightarrow \frac{32}{3} - \frac{20}{3}y - 3y = 1$$
    \pause
    $$ \frac{29}{3}y = \frac{29}{3} \Longrightarrow y = 1 $$
    \pause
    $$ x  = \frac{8}{3} - \frac{5}{3}y = 
      \frac{8}{3} - \frac{5}{3} = 1. $$
\end{center}
}

\makesubsection{Eliminação}
\myframe {
  \begin{center}
  \systemtwo{
    3x & + & 5y & = & 8 \\
    4x & - & 3y & = & 1
  } \spc \\
  \pause
  \systemtwo{
    12x & + & 20y & = & 32 \\
    12x & - & 9y & = & 3
  } \\
  \pause
  $$ L_1 - L_2 = 20y + 9y = 32 - 3 \Longrightarrow 29y = 29 
    \Longrightarrow y = 1.$$
  \pause
  $$ 4x = 1 + 3y \Longrightarrow 4x = 1 + 3 \Longrightarrow 
    4x = 4 \Longrightarrow x = 1. $$
  \end{center}
}

\myframe {
  \begin{center}
  \systemtwo{
    3x & + & 5y & = & 8 \\
    4x & - & 3y & = & 1
  } \spc \\
  \pause
  \systemtwo{
    x & + & \frac{5}{3}y & = & \frac{8}{3} \\
    4x & - & 3y & = & 1
  } \\
  \pause
  $$L_2 \leftarrow L_2 - 4L_1$$
  \pause
  \systemtwo{
    x & + & \frac{5}{3}y & = & \frac{8}{3} \\
      &  & -\frac{29}{3}y & = & -\frac{29}{3}
  } \\
  \pause
  $$ y = (-\frac{29}{3})/(-\frac{29}{3}) = 1. $$
  \pause
  $$ x = \frac{8}{3} - \frac{5}{3}y =
         \frac{8}{3} - \frac{5}{3} = 1. $$
  \end{center}
}

\makesubsection{Gráfico}
\myframe {
  \begin{center}
  \systemtwo{
    3x & + & 5y & = & 8 \\
    4x & - & 3y & = & 1
  }  \\
  \begin{tikzpicture}[domain=-0.1:3.0,scale=1.0]
    \draw[very thin, color=gray] (-0.2, -1.2) grid (3.2, 2.2);
    \draw[->] (-0.2,0) -- (3.2,0) node[right] {$x$};
    \draw[->] (0,-1.2) -- (0,2.2) node[above] {$y$};
    \pause
    \draw[color=red] (8/3,1pt) -- (8/3,-1pt)
      node[below] {$\frac{8}{3}$};
    \pause
    \draw[color=red] (1pt,8/5) -- (-1pt,8/5) 
      node[left]  {$\frac{8}{5}$};
    \pause
    \draw[color=red] plot (\x, 8/5 - 3*\x/5)
      node[below right] {$3x + 5y = 8$};
    \pause
    \draw[color=blue] (1/4,1pt) -- (1/4,-1pt) 
      node[below] {$\frac{1}{4}$};
    \pause
    \draw[color=blue] (1pt,-1/3) -- (-1pt,-1/3) 
      node[left]  {$-\frac{1}{3}$};
    \pause
    \draw[color=blue] plot (\x/2, 2*\x/3 - 1/3)
      node[above] {$4x - 3y = 1$};
  \end{tikzpicture}
  \end{center}
}

\makesubsection{Mais variáveis}

\myframetop{
\begin{center}
  \only<1,2>{
  \systemthree{
    4x & - &  y & + & 2z & = & 4 \\
     x & + & 3y & - &  z & = & 3 \\
    -x & + & 2y & + & 3x & = & 13
  } \\ 
    \only<2>{
    $$ L_1 \leftrightarrow L_2 $$
    }
  }
  \only<3,4>{
  \systemthree{
     x & + & 3y & - &  z & = & 3 \\
    4x & - &  y & + & 2z & = & 4 \\
    -x & + & 2y & + & 3x & = & 13
  } \\ 
    \only<4>{
    \begin{align*}
     L_2 & \leftarrow L_2 - 4L_1 \\
     L_3 & \leftarrow L_3 + L_1
    \end{align*}
    }
  }
  \only<5,6>{
  \systemthree{
     x & + & 3y & - &  z & = & 3 \\
       & - &13y & + & 6z & = & -8 \\
       &   & 5y & + & 2z & = & 16
  } \\
    \only<6>{
      \begin{align*}
        C_2 \leftrightarrow C_3 \\
        L_2 \leftrightarrow L_3
      \end{align*}
    }
  }
  \only<7,8>{
  \systemthree{
     x & - &  z & + &  3y & = & 3 \\
       &   & 2z & + &  5y & = & 16 \\ 
       &   & 6z & - & 13y & = & -8
  } \\
    \only<8>{
      \begin{align*}
        L_2 \leftarrow L_2/2
      \end{align*}
    }
  }
  \only<9,10>{
  \systemthree{
     x & - &  z & + &  3y & = & 3 \\
       &   &  z & + &  \frac{5}{2}y & = & 8 \\ 
       &   & 6z & - & 13y & = & -8
  } \\
    \only<10>{
      \begin{align*}
        L_3 \leftarrow L_3 - 6L_2
      \end{align*}
    }
  }
  \only<11-14>{
  \systemthree{
     x & - &  z & + &  3y & = & 3 \\
       &   &  z & + &  \frac{5}{2}y & = & 8 \\ 
       &   &    & - & 28y & = & -56
  } \\
    \onslide<12->{
      \begin{align*}
        y & = (-56)/(-28) = 2 \\
        \onslide<13->{
        z & = 8 - \frac{5}{2}y = 8 - 5 = 3 \\}
        \onslide<14>{
        x & = z - 3y + 3 = 3 - 6 + 3 = 0}
      \end{align*}
    }
  }

\end{center}
}

\end{document}
